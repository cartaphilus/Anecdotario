\documentclass[letterpaper, 12pt]{book}
\usepackage[utf8]{inputenc}
\usepackage[T1]{fontenc}
\usepackage{calligra}
\usepackage[spanish]{babel}
\renewcommand{\baselinestretch}{2}
\title{{\sffamily \huge Anecdotario de un consultorio veterinario}{\calligra{}Vivencias de una asistente de veterinario.}}
\author{{\Large  MarPerla }\textit{marperla@openmailbox.com }}
\date{\today}
\begin{document}
\maketitle
\tableofcontents
\chapter{La Vocación}
Mi esposo es médico veterinario, y me platica que desde que tiene memoria, le han gustado los animales; yo pienso que tal vez la heredó, pues me cuenta, que su mamá a su vez, le decía que se parecía mucho a su abuelo en esta afición. También me ha platicado que él tendría unos siete u ocho años, cuando su mamá le mostró una foto del abuelo, a quien no conoció, tenía los brazos extendidos, y lleno completamente de palomas, sólo se le veía la cara, algo debió ponerse, para que las palomas se posaran de esa manera. 

El niño quedó muy impresionado con esta fotografía; siempre había tenido animales muy variados, pero a los quince años llegó a tener más de 300 palomas. 

Vivía en un departamento diminuto, de manera que las palomas estaban libres en la azotea: Yo creo que los vecinos lo querían, y mucho. Si no, ¿como aguantarían tanto excremento, plumas y ruido?.

Estas palomas fueron su mayor preocupación a los quince años, cuando tuvo que emigrar del D.F. a León. Tuvo que elegir sólo las que estaban empollando; las empacó en huacales de jitomates, a los que laboriosamente dividió en compartimientos, poniendo a todos los machos juntos en el cajon inferior, cada hembra en un compartimiento, y finalmente formando dos atados con el conjunto.

Volviendo al  abuelo, nos contaba mi suegra que él tenía varios animales, entre ellos un loro que nunca vio volar, y un perro que siempre estaba al lado de él. El día en que murió el abuelo y lo llevaban al panteón, también iba el loro ¡VOLANDO!, y el perro también iba como siempre al lado de su amo. Al loro no lo volvieron a ver, y el perro se echó sobre la tumba, y por más que hicieron por llevárselo, él se quedó ahí. 

Además de la vocación, mi esposo heredó relojes de pared, radios, lámparas etc., ya que estaban abandonados en una bodega, y eran solo cosas viejas que a nadie interesaban; pero cuando nos casamos, mi gordo les dio nueva vida, y como a ambos nos gustan las antigüedades, las conservamos aún, después de 45 años de casados.

\section*{\centering * * *}
¡Bbrrhh! ¡Que frío hace, y no deja de llover ni de día ni de noche. Es enero de 1967, y por fin estoy en la escuela preparatoria de León (``E.P.L.''). Digo por fin, por que se acostumbra en mi familia, que los pasos de las hermanas mayores los sigan las menores.

Estudié la primaria y la secundaria en colegios particulares, siendo muy rígidos, igual que mis papás (Agradezco la vida y todo lo que me han dado), pero me muero por tratar y platicar con muchachos, con naturalidad, sin que esté prohibido. Es por eso que me encuentro muy agitada emocionalmente, y los últimos meses, no he podido casi dormir, por estar esperando esta nueva libertad. Pero me cuido mucho de esconder lo qwe siento, no vaya a ser que a mis papás les parezca peligroso, y no me permitan estudiar en esta escuela.

Mi papá no quiere que estudiemos una carrera corta, aquí en León solo hay Medicina y Contador Público. He elegido Medicina, aunque no tengo intención de estudiar en serio, sólo quiero un pretexto para salir de mi casa, por que ahí encerrada no voy a conocer a nadie.

Estoy soportando el frío aquí en el corredor del segundo piso, junto a la puerta del salón de Química. Como llegué un poco tarde la puerta ya está cerrada, y no me atrevo a pedir permiso para entrar, soy muy tímida y no soporto las miradas en mí, estoy deseando que llegue alguna compañera, para entrar juntas; la maestra no nos negará la entrada, se llama Arcelia Chávez Fonseca, y es madrina de una de mis hermanas menores.

Las clases comenzaron en enero, porqwe están recorriendo el calendario escolar, pues quieren que comiencen en agosto, y no en febrero, como ha sido toda mi vida, ¡Que raro!.

Estoy aquí de pie, con un libro abierto, para disimular que me siento sola y tonta. Este libro lo muevo a mi antojo para ver quién pasa.
 
Y desde este lugar alcanzo a ver la entrada de la escuela. ¡Oh, acaba de entrar un muchacho que desde aquí parece guapo, tiene figura atlética, ojalá que pase por aquí!

¡Dios mío, el corazón se me acelera, ya subió hasta este piso, y viene en mi dirección.! Muevo el libro abierto, y con un solo ojo veo que viste con buen gusto. Me tapo con el libro, y oigo más fuertes los latidos de mi corazón, se viene acercando, subo un poco el libro, y puedo ver sus zapatos boleados, a pesar de la lluvia, y los calcetines muy de acuerdo con el color de la ropa. Decido acercarme más a la puerta para que le sea inevitable verme y saludarme. ¡Me he quedado pasmada!. Con toda naturalidad y muy seguro de sí toca, y por la ventanita redonda de la puerta levanta un dedo pidiendo permiso y entra, dejándome su aroma de recién bañado y sin haber notado mi presencia: ¡Me siento frustrada!

\section*{\centering * * *}
Aquel muchacho guapo de enero lluvioso y frío, no lo he vuelto a ver, y ya hasta lo olvidé: Esto se debe a que en el grupo de medicina en el que estamos ambos, es muy numeroso, consta de cien personas, de las cuales solo el diez por ciento somos mujeres.

Los maestros exigen que primero entremos las muchachas, y ocupemos los pupitres de adelante, y después los muchachos, donde quieran sentarse: Yo soy incapaz de voltear hacia atrás. Igualmente al concluir la clase, primero salimos las mujeres y después los hombres, cambiando de salón para cada asignatura. Pero mientras llega el siguiente maestro, podemos platicar muy a gusto. ¡Que diferencia de las otras escuelas!. 

Yo estoy platicando con mis compañeras, pero siento la mirada de varios muchachos.

\section*{\centering * * *}
Ya es marzo; el prefecto entra al salón de biología, al concluir la clase, y nos indica que debemos elegir representante de grupo. Me siento muy sorprendida y angustiada, quiero con todas mis fuerzas desaparecer, no se porqué desgracia comienzan a decir mi nombre, y cada vez se unen más voces. Yo siento tanto ardor en la cara, que creo que me va a estallar.

Declino la invitación, digo que no se cuantas cosas me lo impiden; el muchacho guapo de enero, ahora lo veo frente a mi, pidiéndome que acepte; yo quiero dirigirme a la puerta, pero me cierran el paso. Por fin logro convencer a mis compañeros, y eligen en mi lugar a un muchacho:¡Que alivio!.

Comienzo a darme cuenta que en cada cambio de salón el muchacho guapo está en el mesa-banco justo detrás del mío, y así continúa, a corta distancia de mí. Veo que diario juega ``frontón'', y anda con un grupo de amigos, haciendo bromas, divirtiéndose y entorpeciendo las clases. Los han llegado a sacar del salón algunos maestros; son bromistas e inquietos, pero observo que no son groseros.

Pasan los días, y comenzamos a platicar, ya sé que se llama Armando Antonio;a mí me parece incómodo: lo siento muy largo. Desde la secundaria todos lo conocen como \textit{El Bailarín}.

Después de algunos días, me pide acompañarme a mi casa al terminar las clases, dejándome en la esquina. 

\section*{\centering * * *}
Llegan las vacaciones de Semana Santa, mi papá nos lleva a Mazatlán, y yo estoy pensando mucho en aquel muchacho, deseo que terminen las vacaciones para volver a verle. Por casualidad escucho varias veces ``L'amour est bleu'' (acá conocida como ``El amor es triste'') con Paul Muriat. 

Cuando volvemos a vernos, me dice que él también salió, se fué a  México, y que le llegó fuerte la canción ``Celoso'', con Marco Antonio Muñíz, la cual escuchaba en una grabadora, regalo de su tía que vive allá. 

\section*{\centering * * *}
¡Ya ha llegado abril, me parece glorioso e idílico, me siento enamorada. El día 7 me pregunta si quiero ser su novia; yo contesto que necesito tres días para pensarlo, debo reprimir a todas mis células que quieren gritar ``¡Siiiii!''.

Lunes 10 de abril, salimos del laboratorio de Química en el 3er.\ piso con el ingeniero Aranda Ulíbarri, es la última clase, y me pregunta qué he decidido, yo contesto que sí. Siento que varios músculos de la cara se mueven involuntariamente.

Salimos de la escuela y con toda naturalidad me pasa un brazo por los hombros y me besa en la mejilla. Siento que me deshago, me percibo torpe para caminar, no puedo articular palabra, me aturde el roce y el aroma de ese cuerpo joven y masculino.

Me acompaña hasta la esquina de mi casa, que está muy cerca, llego y me voy a mi recámara, para que nadie vea mi agitación.

Nos vamos conociendo y me siento más y más enamorada; el carácter de este joven alegre, desinhibido, sociable y capaz de todo lo que se propone, me abre la puerta a un mundo hermoso, donde siempre reímos, y del cual yo no quiero salir.

Para alargar nuestro tiempo juntos, damos vueltas al jardín principal, o nos sentamos en alguna banca, contemplando las palomas, y soñando\ldots.

Yo sueño con estar todo el tiempo con él, sin el sobresalto de que nos vayan a ver y me regañen.

Puedo afirmar con toda seguridad que esta etapa de enamoramiento es hermosísima, y que todos tenemos derecho a vivirla para guardarla por el resto de nuestra existencia. Y sin esta experiencia, nadie debería casarse.

Mientras cursamos la preparatoria, me siento muy feliz, con derecho a ser joven, y disfruto mucho las alegres e inocentes tardeadas, a las que me voy a escondidas, y que se llevan a cabo en el club de Leones, o en el Rotario, o en cualquier casa, me da lo mismo, me gusta estar con personas de mi edad, escuchando la música del momento. 

Mi novio sabe mucho de música, y además baila de todo (Incluido el  ``ago-gó'', que está de moda junto con los colores ``sicodélicos'' que usamos).

Los martes y jueves el cine cuesta la mitad, y acudimos con frecuencia. Nos vamos adhiriendo uno al otro cada vez más.

En la grabadora de Tony escuchamos mucho a ``The Beatles'', y esa balada de ``When I'm sixty-four'' (``Cuando tengamos 64'') nos causa mucha risa, al imaginarnos de esa edad. Nos parece inconcebible pensar en llegar a tener más edad que nuestros papás, ``No, no puede ser, falta toda una vida\ldots''.

Estoy escuchando en la sala de mi casa, un disco de los hermanos Castro, que me prestó Tony, el cuál dice que es de sus favoritos. Por mi cuenta, me gusta escuchar a los hermanos Carrión, Enrique Guzmán, Roberto Jordán, Rocío Dúrcal, etc pero prefiero estar a solas, para que nadie critique mis gustos.

Todas las mañanas al salir de la casa para ir a la escuela veo a Tony que me espera en la esquina, y me entrega un poema que yo voy guardando en una libretita. No es que tenga alguna veta de poeta, sino que junta un trozo de una canción, otro de otra, lo que más fuerte le llega, lo acomoda y me lo dedica; me llama ``Muñequita''; a mí me importa mucho lo que siente y cómo me lo expresa, lo que me indica que también está enamorado.

Los compañeros lo ``agarran de bajada'', preguntándole por qué ya no anda con sus amigos; yo me sonrojo, no me gusta que piensen que lo acaparo, lo que pasa es que nos atraemos mucho. Mis amigas también se han alejado, por que no tengo tiempo para ellas. 

Ahora que empiezo el segundo año de preparatoria, mi único hermano varón ha entrado a la secundaria en este mismo plantel, llevando la consigna de no volver a casa sin mí. Me resulta muy fácil negociar con él: Le pido que nos acompañe a la deportiva y él a cambio pide un cuento de \textit{Rolando el Rabioso}, o de \textit{Memín Pinguín}. Se vuelve muy común que andemos paseando por el centro y mi hermano atrás leyendo su revista.


\section*{\centering * * *}
Ya estamos presentando exámenes finales, y algunos maestros los hacen orales, y con tres sinodales, esto nos causa pánico, preferimos que sean escritos. Hay mucho alboroto entre los compañeros, hablan de continuar los estudios fuera de León, solo una minoría es aceptada aquí.

En León no hay Medicina Veterinaria, y desde luego lo que más le conviene es la UNAM, porque cuenta con la casa de sus abuelos maternos, y con su tía para asistirlo\ldots

Hoy se va Tony a México, cómo hemos temido este día, le prometí que iría a despedirlo a la terminal de autobuses ``Estrella Blanca''. Vamos caminando como viejos, cargamos sobre nuestros hombros una gran pena; él dice que no sabe cuando vendrá, que no tiene dinero. Nos prometemos escribirnos cada semana, se va el camión, él está llorando y contemplándome por la ventanilla hasta el último momento. Yo me vuelvo a mi casa, sintiéndome devastada, no comparto mis sentimientos con nadie; me siento tan sola, no siento confianza con nadie, solo me siento consolada con mis hermanos menores. 

¡Ya llegó la primera carta!, pero antes debo leer el instructivo, el cual dice que lea con un espejo; calculo que él también tiene ya la mía. 

Para leer la segunda carta hay que rotarla constantemente, pues cada línea continúa en la cara opuesta de la hoja. Estoy esperando la siguiente carta, cuando me dicen que me llaman por teléfono, con gran sorpresa y felicidad, escucho que es él. Me cuenta que sus hermanas telefonistas, se las arreglaron para comunicarnos, y que lo seguirán haciendo: ¡Muchas Gracias!!.

La tercera carta, debo empezar a leerla por el centro de la hoja y seguir en espiral. Así continuamos cada semana, una carta y una llamada. 

\chapter{Afrodita Enamorada}
Cuando Tony y yo eramos novios, en una ocasión me llevó a su casa, y conocí a su perrita Afrodita. Era una perra de talla muy chica, y china, tal vez mezcla de \textit{poodle}, y \textit{lhasa apso}. 

Esta perrita estaba educada para no subirse a los muebles, solamente podía usar su camita; comía únicamente albóndigas pequeñas, que mi suegra tenía la paciencia de prepararle; y no tomaba agua, sino ``7Up'' (Conocido refresco sabor lima-limón).

Como sabemos, las perras son fértiles cada seis meses, y nadie sabe cómo se hizo novia de un perro grande, carlangudo, que muy puntual cada seis meses estaba junto a la puerta---y se trataba de un departamento en planta alta, de donde no parecía haber oportunidad de salir. Aún así tuvo varias camadas de este perro---. Le llevaban perros afines a ella en estatura y fisonomía, pero todos eran rechazados por ella. Sólo el grandote le gustaba.

Tony, curioso, un día lo siguió para saber de donde venía; cruzaron tres barrios, equivalente a unos 3 kms.\ y el perro continuaba, así que mi gordo desistió, sin llegar a conocer su morada.

Cuando terminamos la ``prepa'', mi entonces novio, se fué de León, para estudiar en la UNAM, y sin saber cuándo podría venir; De vez en cuando reunía dinero para el pasaje, y tomaba el autobús de media noche, y amanecía aquí sin haber avisado a nadie. Pero mi suegra ya sabía que venía su hijo, porque desde la media noche, Afrodita lloraba y gemía junto a la puerta, olisqueando por debajo, hasta que llegaba su amigo predilecto, recibiéndolo con saltos y muestras de mucho cariño.

Difícil de creer, pero así ocurría\ldots

\chapter{Veterinario \textit{Express}}
Tony fué a México a hacer el examen de admisión para la universidad, y llegó allá precisamente la noche del 2 de Octubre de 1968, se dirigía a la colonia Morelos, muy cerca de Tlatelolco. Iba a vivir en la casa de su abuelo ---el de las palomas---. Allí radicaba su tía, hermana de su mamá, que era soltera. 

Al ir acercándose, preguntó al taxista cuál era la razón del festejo, ---¿por que tantos \textit{cuetes}?---¡No joven, es el ejército que está atacando a los estudiantes, está la cosa que arde!---.
Al llegar a la casa, encontró a su tía alarmadísima, y le dijo ---Qué bueno que te dejaron pasar, no sé que está ocurriendo, pero en la televisión se nota mucha alarma, y desde la tarde escucho tiros aquí, en Tlatelolco---. Estas fueron las primeras noticias que recibió de ese triste episodio de nuestra historia. 

Era un caos, la universidad estaba cerrada, y a Tony le tocó hacer su examen de admisión en el estadio de Ciudad Universitaria. 

Pocos meses después, cuando se dio cuenta de que había sido aceptado, dice él que ya se sintió veterinario. Vino a León y me pidió en matrimonio, pues también a mí me era muy difícil estar lejos de él; pero mi papá puso como plazo un año, que nos pareció interminable.

\chapter{Nuestra boda}
Estoy muy nerviosa, acordamos Tony y yo que hoy viernes 9 de  marzo de 1969, vendría con sus papás y un sacerdote amigo (el padre Herrera), a pedir mi mano: Queremos casarnos en dos o tres meses, para no estarnos extrañando.  Tengo miedo, no sé como reaccionará mi papá; a mi mamá ya le avisé, y cosa rara, lo tomó con mucha calma, me preguntó qué prepararía yo, y le he dicho que tengo la receta de un pastel de cuadritos de chocolate y vainilla; ya he escuchado que para quedar bien se deben servir alimentos ya experimentados y no correr el riesgo de no tener nada para el momento; yo no soy aficionada a la cocina y no sé ni lo esencial, pero tengo mucha confianza en que saldrá bien; de una vez lo voy a hacer para que no me tomen las prisas\ldots

Dejé el pastel horneando y subo a mi recámara para elegir el vestido que me pondré, al entrar siento como una bofetada la acritud de una de mis hermanas que me dice ---``Tanto que se ha esforzado mi papá por darnos una buena posición, para que tú te cases con un muchacho pobre y de apellido sin ningún lustre\ldots''---. Yo no soy capaz de contestar nada, a pesar de estar tan enamorada, creo que tiene razón, pero nada me va a detener. 
 
Hasta este momento caigo en la cuenta de que no hemos hablado de ``de qué vamos a vivir'', pero pienso que no debe ser importante: Lo verdaderamente importante debe ser estar juntos.

Ya casi es hora de comer, y yo no tengo hambre, además temo mucho estar a la mesa con mi papá. Ya saqué el pastel y quedó bien, dejaré que se enfríe para desmoldarlo.

Ya son las 8:00 p.m., Tony me dijo que vendrían a las 8:30; apenas puedo sostener el cepillo para arreglarme el cabello; no quiero demostrar mi agitación interna, porque temo las burlas de mis hermanas mayores, o sus reproches.

¡Ya sonó el timbre, ya llegaron!. Hace un rato mi mamá nos dijo que ya había informado a mi papá, quien se metió a bañar enseguida. También nos advirtió que este era un asunto sólo de ellos dos: Que ya nos llamarían cuando tuviéramos que bajar\ldots
Me acerco mucho a la sala con el oído alerta, pero no escucho nada, así que regreso y me quedo sentada en el último escalón para oír cuando me llamen. Ya pasó media hora; ¿aceptaría mi papá?\ldots Por lo menos no los ha corrido. Yo soy su primera experiencia con esto de la ``petición'', por que a mi hermana Yola, la mayor, no la pidieron, simplemente desapareció una noche, hace 13 años.  Al día siguiente se presentaron los papás del novio, diciendo que ella estaba bien y pronto se casarían.

¡Oh, me está llamado mi mamá!. No logro controlar el temblor de mis piernas, temo caerme, los visitantes me ven bajar:¡Oh, Señor, que no me caiga!.

Veo a Tony que está sentado con todo su aplomo, sereno. Comienzo a serenarme.

¡Que decepción! Mi papá me ha informado con toda cordialidad que ha puesto un año de plazo, dice que yo no sé nada de las labores del hogar, pero me pasa la idea de que lo que realmente le interesa es que estemos seguros de lo que vamos a realizar. Yo sé que no debo opinar ni discutir, ya puedo sonreír y me voy a la cocina con mi mamá para servir la cena, la cual termina pronto debido a que como no se conocen, hay cierta tensión. El pastel, milagrosamente, estuvo a la altura de las circunstancias. Ya para irse sonó el timbre y Tony me dice que aún sin conocer el resultado de la entrevista, invitó a un reportero, el cual informará a la sociedad sobre este evento: desde ahora dejamos de ser novios para pasar a ser \textsc{prometidos}.

Respecto de mis hermanas mayores, por estar estudiando me les adelantaría; esto no les gusta nada ya que circula el refrán ``Chivo brincado, chivo quedado\ldots''.

Durante este año de plazo ya no quiero ir a la escuela, a pesar de la insistencia de mi papá. Yo solo quiero vivir para mi futuro esposo y mi hogar, pienso tejer carpetas, bordar sábanas, aprender a cocinar, lo cual intento con mi mamá, pero se la pasa regañándome, mejor me voy a inscribir en el \textsc{imss}.

Ya es noviembre de 1969, y Tony llegó de México, me pregunta qué opino sobre casarnos al civil el día 22, a lo cual accedo con mucho entusiasmo; voy a informarle a mis papás y les explico que ese es el día de Santa Cecilia y del músico, lo cual nos gusta porque Tony desde muy niño es melómano.

Tony invitó a sus testigos y yo a los míos. Uno de ellos era una amiga desde la primaria, pero al saberlo mi papá, no estuvo de acuerdo, dijo que debían ser personas serias y con más edad; mejor mis testigos serían sus amigos. Mi papá, aunque muy estricto, también es alegre (a él le debo mis primeros pasos de baile). Es sociable y tiene muchos amigos, pero en este momento a quienes veo con más frecuencia es al sr. Ramiro Cruz, a su mamá, sus hermanos y su esposa Tere; al sr. Enrique García Bueno y su esposa Ma. Liduvina, y su amigo de la juventud, el sr. Tobías Aguilera, quien reside en Moroleón.

Elijo para esta boda un vestido de ``fayuca'': en esos momentos es de mejor calidad que lo del pais. Es verde botella, con cadenas doradas al frente, tiene manga larga, la falda es en linea A y muy arriba de la rodilla, como se usa ahora.

La sala ya está llena con los invitados y los testigos; a las siete en punto llega el Lic. Aranda Guedea y requiere ante él a la pareja de novios; nos dirige un discurso, luego pide las firmas y da por concluida la ceremonia al entregarme la ``factura'' de mi marido.

Las posadas, la Navidad y el Año Nuevo quedaron atrás; ya es enero de 1970, y llega Tony de México; estamos platicando, saca un calendario y me señala el 14 de febrero, que es sábado; yo lo veo con interrogación y él me pregunta que si me gustaría esa fecha para nuestra boda religiosa: Yo brinco de alegría y lo abrazo, diciendo que no hay mejor fecha. Seguimos platicando sobre como queremos las invitaciones, las cuales las harán unos vecinos suyos.

A los pocos días ya están hechas, me encantan, son de papel pergamino, que entregaremos enrolladas; tienen un aire tan antiguo (¿me gusta lo antiguo?).

Vamos exultantes a mostrárselas a mi papá, quien al leer, pregunta muy serio: ---``¿Porqué se adelantan?, quedamos que un año''---. Al oírlo me quedé fría, no pude contestar nada, sentí que mi sangre helada se iba muy lejos de mí. Pero me volvió el alma cuando él mismo agregó que pasaría por alto esto de la fecha, al fin solo se trataba de tres semanas.

Fueron mis hermanas mayores quienes decidieron que mi vestido, el de ellas y el de mis cuñadas ---ya que todas ellas me acompañarían como madrinas---, los hiciera una señora de Silao, y allá íbamos a las pruebas.

Estos vestidos de las madrinas me gustan mucho, son de terciopelo y evocan el Medioevo. (Si, creo que me gusta lo antiguo).  
Mi vestido sigue la moda del momento: Es en línea A, de organza con aplicaciones de encaje y bordados de lentejuelas transparentes y nacaradas; me gusta como brillan: Con la luz eléctrica van a brillar mas.

Mi mayor hermana soltera está decorando mi ramo, y ella lleva la logística de mi boda. También me ha comprado un ``babydoll'' pequeñito y transparente.

Desde que nos casamos por lo civil Tony y yo hemos disentido varias veces. Él insiste en que yo use su apellido, pues legalmente ya soy su esposa; yo me resisto diciendo que esta otra boda es la importante, a lo que él contesta que si lo desea, me podría llevar con él. Un día, algo fastidiada por estas discusiones le digo que tiene razón en todo, que solo le pido que informe a mi papá de todo esto que él me dice: Creo que no hemos vuelto a discutir al respecto\ldots

Cada vez que él viene de México, vamos a repartir invitaciones, y yo me siento mucho muy emocionada, conociendo a sus familiares, que pronto serán también los míos; y yo lo llevo con mis familiares, a quienes fácilmente les resulta simpático.

Hemos elegido el templo de Nuestra Señora de Los Ángeles, el cual está a cargo de nuestro maestro de etimologías grecolatinas. Nos pregunta si nos parece bien que nuestra misa comience a las 6:30 p.m., a lo que le contestamos que no tenemos ningún inconveniente; continúa diciendo que entonces ya no tenemos que pagar nada, porque antes y después de la nuestra hay otras bodas, y ya han pagado todo. Nosotros comprendemos que nos está ayudando, y conmovidos nos despedimos.

Ya amaneció el día 14 de febrero, y se siente mucha agitación en la casa, todos andamos de prisa, me llama el maestro de música para avisar que no le es posible acompañarnos con su canto; yo no sé que hacer, y al poco rato otras cosas reclaman mi atención, y me olvido del asunto.

La peinadora y la maquillista llegarán a las 2:00 para arreglar a todas las mujeres de mi casa. Yo soy la primera en bañarme y ya tengo la maleta lista.  A las 5:00 Tony ha mandado un auto güinda, último modelo, adornado con flores blancas, el cual han metido hasta la puerta del comedor, por donde debo salir.  Mi mamá se ofrece para ayudarme con mi vestido; tiene muchos botoncitos en la espalda, y al empezar a abotonarme escucho en el radio a ``Los Archies'' cantando ``Sugar, sugar''. Estoy tan emocionada que me pongo a bailar, sintiéndome reina, y mi mamá me suplica que me quede quieta, si no, no podrá terminar.

Ya debo bajar, porque en la fotografía nos esperan antes de la misa. Bajo y una de mis hermanas les habla a mis papás, yo me arrodillo ante ellos y les pido su bendición; mi hermana toma fotos, y también frente al coche. Tony me está esperando en la fotografía Dávalos, me ayuda a bajar del coche, admirando a su novia con el atuendo de boda, me dice muchos piropos, y yo me sigo sintiendo una reina. ---¡Él está tan guapo con su traje negro!---.

Al llegar al templo, me sobresalto al recordar al maestro de Música, se lo cuento a Tony, y él a su vez se mete a decírselo al padre, quien ya con sotana sale corriendo, va a Catedral y regresa con músico y cantor. ¡Gracias, padre Zarate!.

Vamos siguiendo al padre por el pasillo de enmedio, yo del brazo de mi papá, y extrañamente no siento vergüenza, me siento reina. Con las luces, mi vestido brilla hermosamente ---creo que estos destellos nacarados no los voy a olvidar nunca---. Lo mismo el lazo y mi rosario, que son de pedrería.

Algunos de mis invitados que llegaron tarde y no nos vieron entrar, al escuchar al padre diciendo: ---``Así como Martha ama a Armando y Armando ama a Martha\ldots''--- se preguntaron ---``¿Siempre no se casó con Tony?''---\ldots

\chapter{Perros Delincuentes}
Cuando mi \textit{gordo} iba en el tercer año de la universidad, en los alrededores de la casa ya lo conocían, y comenzaban a llevarle sus animalitos.

Había unos muchachos que iban con frecuencia, y llevaban perros \textit{doberman} heridos. Él los atendía, quedaba intrigado, pues los perros siempre llegaban con cortes como de arma blanca; les preguntaba, y ellos contestaban:---``Es que los peleamos por apuesta\ldots---. Generalmente traían dinero, y eran generosos, pues le pagaban más de los que el les cobraba.

Recuerdo muy bien que ya le llamaban ``doctor'', y yo lo oía y sentía muy bonito.

En una ocasión, cuando ya teníamos nuestro primer coche, --- viejito, pero nos llevaba y traía a León --- Estábamos contemplándolo muy tristes, porque había amanecido sin limpiadores, sin los tapones de los rines, etc. Cuando llegaron los muchachos, y preguntaron ---¿Que pasó doctor?--- el les enumeró las cosas que le faltaban al coche; Ellos le dijeron---Espérenos un momento---, y al poco rato, regresaron con todo lo que faltaba, preguntando si estaba todo completo. Dijimos que sí, pasmados, sin entender, pero agradecidos, y no volvió a ocurrir, nunca nada semejante.

Este cochecito era un Opel 62 y nos lo ofreció en venta un querido amigo. Mi \textit{gordo} al instante lo aceptó, pensando que en una o dos semanas podría pagarlo, y hasta le sobraría, porque acababa de comprar una camada de \textit{doberman}, y eran 11 cachorros, que podía vender en 150 pesos cada uno. 

Los perritos comenzaron con diarrea, luego diarrea con sangre. Corrió con su maestro el dr. Alfredo Cortez a consultarlo, y fué así como conocimos la terrible enfermedad conocida como ``parvovirus''. También le informó que esta enfermedad había aparecido recientemente en México, y no había medicamento eficaz; y que debía tratarla sintomatologicamente, o sea, que si había diarrea, combatiera la diarrea; si se presentaba deshidratación, tratar la deshidratación.

De modo que fuimos perdiendo uno por uno los once cachorros, y lo que íbamos a pagar en dos semanas, lo hicimos durante todo un año.

Agradezco la paciencia y generosidad de nuestro amigo\ldots

Tiempo después dejaron de ir los muchachos. Por la televisión nos enteramos que habían atrapado a la banda de los \textit{doberman}; luego algunos vecinos nos informaron que habían asaltado, utilizando perros, a un sobrino del presidente Luis Echeverría, y al caso se le dio seguimiento hasta dar con la ``cabecilla'', que era apodada ``La Jitomata'', la cual tenía dominada a la colonia Morelos. 

Por orden presidencial fueron apresados todos, a pesar---después se supo--- de haber ofrecido 2 millones de pesos por su libertad, los cuales no les fueron aceptados. Los perros fueron destinados a la cámara de gas.

Estábamos temerosos y pensando en escondernos, porque no sabíamos si al doctor lo considerarían involucrado por haberles atendido a los perros; y como comprendimos después, a los perros no los ponían a pelear, como nos habían platicado, sino que los azuzaban contra las personas para asaltarlas, y estas se defendían como podían. De ahí las heridas que siempre presentaban.

Ya después que tuvimos más información sobre esta banda, nos llenamos de terror, al recordar que en una ocasión, los muchachos se llevaron un cachorro que mi esposo tenía en venta, para mostrárselo a alguien, prometiendo traer el dinero en seguida.

Como pasaron tres días, y ni cachorro ni dinero, Tony, molesto, se informó en dónde vivían y fué a meterse en una vecindad enorme, donde vio gente dormida en el piso por todos lados.  El llegó y exigió su dinero o su perro, el cual le pagaron en seguida, pidiendo disculpas.

Al contarle esto a un vecino y amigo, este lo reconvino por haber hecho lo que hizo, y le contó a su vez cómo, en una ocasión en que alguien fué a reclamarle algo a los mismos, fué recibido con un hachazo en la frente; platicándole también que alguien no cumplió un acuerdo, y se presentaron en su casa, aniquilando a toda la familia.

¡Que suerte tuvimos\ldots!

\chapter{Aprendiendo a cortar orejas}
Por fin estábamos juntos, ocupando el fondo de la planta baja de aquella casa tan grande y antigua. En el piso de arriba vivía su tía; el piso de más arriba, y el resto de la casa, estaban rentados. Muy pronto, con poco dinero y gran esfuerzo, fuimos convirtiendo nuestra residencia en un rinconcito de Guanajuato (en pleno Distrito Federal).

De la bodega donde se guardaban los tiliches, iban saliendo que un farol, que un reloj de pared o de mesa, etc.

Mi Gordo trabajaba en la editorial ``Aguilar'', utilizando las horas libres de la universidad; los sábados se iba por el rumbo de Ciudad Satélite, tocando puertas y ofreciendo vacunas o baño para las mascotas, dejando su número telefónico; así fué formando sus primeros clientes.

Un día llegó Tony con su querido amigo Pepe, acompañados por un perro que habían encontrado en la calle, al cual convencieron de seguirlos con unas salchichas.

Lo llevaron a la casa porque iban a aprender con él a cortar orejas\ldots

De modo que lo subieron, o mejor dicho, mi gordo lo subió a la mesa de operaciones, porque Pepe les tenía pánico a los perros y sólo iba a intervenir cuando el perro estuviera completamente dormido.

La dicha mesa de operaciones era el primer descanso de las escaleras, este descanso le llegaba a la cintura, era de cemento, y ahí cambiaba de dirección la escalera, la cual fué hecha por el abuelo de las palomas, quien fuera herrero de profesión. Esta escalera llegaba hasta el tercer piso, y repito, era toda de fierro, escalones y pasamanos, cambiando varias veces su dirección. Me dijeron que cuando vivía los abuelos, esta escalera se veía hermosa, porque estaba toda cubierta, hasta el tercer piso, por una enredadera conocida como ``Copa de Oro''; también me contaron que a esta casa, por su altura, le llamaban ``El rascacielos del sr.  Roa.''. 

Me imaginaba  esta casa en todo su esplendor, cuando vivían los papás con los tres hijos. Yo podía ver candiles, lámparas, radios, muchos relojes (dicen que había un en cada habitación), muebles importados y adornos valiosos.

Esta casa medía 60 mts.\ de fondo, y yo creo que la sala estaba en la primera habitación, con ventana hacia la calle, después seguiría el comedor, después la cocina, en seguida el baño que vendría quedando hacia la mitad de la casa. Aquí terminaba el pasillo, que venía desde la puerta de la calle, luego seguía un patio, y en medio de él, como dominando todo, la mencionada escalera, cuyo primer tramo que incluía el mencionado descanso ---``El quirófano''---, era de ladrillo enjarrado, el cual fué hecho por mi esposo al tener que desechar esta parte de la antigua escalera.

Al terminar el patio, y ya en el fondo de la casa, seguía una recamara, espaciosa e iluminada por un gran ventanal, que debe haber sido la alcoba de los señores; arriba de esta, estaba la alcoba de las dos hijas, y más arriba, la alcoba del hijo.

Solo el piso de hasta arriba, se conservaba original; al de enmedio le habían agregado un comedor y una cocina, y al de la planta baja le aumentaron un comedor y un baño grande completo con una gran ventana. Estas añadiduras deben haberse llevado a cabo por los años 50's. 

Nosotros habitábamos en la planta baja, y contábamos con la mitad del patio, hasta donde se encontraba la escalera. Apoyada en esta y en la barda, mi gordo puso una puerta baja de madera, también hecha por él --- Le gustaba la carpintería, y ahorrar, por que el dinero siempre faltaba ---; la pintó con \textit{chapopote}, y para tocar le puso una manita de hierro con una bola, que encontró en la bodega.

A todo lo largo del pasillo y el patio había variadas plantas. 

También debo agregar que estos tres pisos del fondo estaban decorados por unas jardineras de fierro, las cuales fueron hechas por el abuelo, en las que había tres o cuatro macetas. 

Dicha escalera era el límite, sin incluirla, de nuestra área domestica; solo la incluíamos cuando necesitábamos la mesa de auscultación

Y volviendo a lo de las orejas, ya tenían todo listo, y el perro resultó muy dócil, (En la universidad no enseñan a inyectar, cada quien aprende por su cuenta).

Decidieron rasurarle una mano, para que fuera más fácil la inyección intravenosa. Pero no tuvieron suerte, hubo extravasación; entonces le rasuraron la otra, y tampoco tuvieron suerte, luego las patas, logrando inyectarle un poco de anestesia cada vez. Luego decidieron  irse a la yugular, logrando inyectar el resto. Claro que después de cada piquete se ponían a platicar amenamente, dando tiempo a que hiciera efecto la inyección. Lo iban a ver de rato en rato, y el perro estaba dormido, pero de aburrimiento. En fin, el día se terminó, la noche encontró a los aprendices frustrados e impotentes\ldots.

\chapter{Generosidad}
Mi esposo con su carácter alegre, sociable y desinhibido, se gana la simpatía de muchas personas. Acercándose a los maestros más brillantes y compartidos, les hacía muchas preguntas, ya que quería aprender y practicar lo antes posible; necesitaba ingresos económicos. Guarda mucho agradecimiento a varios maestros, pero el que más destaca, es el doctor Cortez.

Este querido galeno, además de contestar a todas sus preguntas, un buen día puso su quirófano a su disposición, y no era cualquier quirófano, ya que este médico era nada menos que el director del Centro de Cirugía Experimental, hoy Siglo XXI, y no sólo el quirófano, también incluía: enfermeras, instrumental médico, medicamentos, y su asesoría. Con todos estos regalos, sus conocimientos y su seguridad subieron como la espuma.

Al contar con este gran respaldo, pudo ofrecer sus servicios, anunciándose en el periódico. Por este medio lo contactó el señor Daniel Pérez Arcaraz, que conducía el programa de televisión ``El Club del Hogar'', junto con \textit{Madaleno}, el cual pasaba por el canal 4. El sr. Pérez Arcaraz quedó muy complacido con la atención que le dio a sus perros, y lo invitó a su programa, para anunciarse también ahí. 

Fue así como lo llamó la China Mendoza, señora de la política, y además escritora, que poseía una perrita \textit{poodle}, que nunca había aceptado a ningún macho, y Tony le ofreció el nuestro, para poder observar el comportamiento de la perrita, pero debían ir a la casa del macho, porque es mejor que éste permanezca en sus dominios. Así lo hicieron, y fué amor a primera vista para ambos.

Después de algunas camadas, la perrita se extravió, y la señora, inconsolable, a través de los medios de comunicación pidió ayuda, prometiendo que daría una recompensa a quien encontrara a ``una niña disfrazada de perra'' ---Así lo expresó---, y sí la encontró. Poco tiempo después la señora cambió de residencia, y los perros no se volvieron a ver. Pasaron cerca de dos años, y volvió la señora con su perrita, y ambos perros se demostraron mucha alegría, ya que los dos seguía fieles y amándose.

Volviendo al doctor Cortez, mi esposo tiene el gusto de saludarlo con cierta regularidad, y recordar su generosidad. Cada año en la primera semana de septiembre se lleva a cabo aquí en León el Congreso Internacional Veterinario; algunos de los conferencistas que conocen al doctor, le informan a Tony como está de salud, y el anota su nombre para que le lleven sus saludos\ldots

Es octubre del 2015, a las 9:30 p.m.\ entro a la cocina y encuentro a mi esposo con el rostro resplandeciente, hablando por teléfono, haciéndome señas y apuntando al aparato; por fin entiendo que habla con el doctor Cortez. Esperé como 20 minutos, viendo cómo disfrutaba al hablar con su maestro.

Cuando terminó la conferencia, la emoción lo embargaba. Casi gritando, me contó que lo había buscado en su casa al medio día, explicando a su esposa quién era y porqué lo buscaba. Me explicó que la señora también estaba emocionada; me informó que el doctor tiene la misma rutina desde hace 45 años: Da clases en la Universidad, y por la tarde en Siglo XX1 continúa con cirugía experimental, agregando que vuelve a casa entre 9:00 y 9:30 p.m-; mi esposo añade que lo más pronto posible lo buscará en ese horario.

Nunca imaginó que después de unas horas, el mismo doctor lo llamaría.

Platicando con el doctor, le recordó cuanto le debía; a su vez el doctor le dijo que nunca lo ha olvidado y que recuerda muy bien sus llamadas telefónicas el día del maestro, y que por cierto, era de los pocos que así lo hacía.

Me repitió todo lo que platicaron, y con mucho énfasis dijo que no le ha cambiado la voz, que sigue igual de activo, y que lo admira mas.

El doctor le hizo una promesa: Como tiene una hermana aquí, y hace tiempo no la ve, vendrá pronto, y hará dos visitas\ldots
¡Infinitas gracias, doctor Cortez!\ldots
\chapter{Un hermoso \textit{poodle} negro}
Nuestro \textit{poodle} que menciono líneas arriba, llegó a nosotros siendo ya adulto; a pocas puertas de nuestra casa, había un taller, y tenían allí al perrito amarrado. Cada vez que lo veía Tony, sentía lástima por verlo así, y observaba la buena calidad del mismo, pero sobre todo le gustó su color negro. Un día se atrevió a preguntar si se lo venderían, a lo cual contestaron que sí, ya que no tenían espacio para el. También le dijeron que se llamaba ``Pelusín''. Muy contento llegó a la casa con el perrito, diciendo que necesitaba con urgencia un baño.

Al estarlo bañando, ¡oh sorpresa!: El perro era tan blanco como la nieve, y efectivamente muy hermoso. Se adaptó muy fácil a nosotros, y nuestro niño mayor  jugaba mucho con el. En una ocasión entró el niño llorando, y diciendo que lo había mordido ``Pelusín''. Extrañado, mi Gordo salió a ver que pasaba con el perro, y le vio una jeringa clavada en el lomo; preguntó al niño si el había hecho eso, y el niño dijo que sí, que el perro necesitaba una vacuna.

No tenía más de tres años, y ya iba tras los pasos del papá; Yo observaba cómo la genética sigue su camino a través de las diferentes generaciones.

Debo agregar que este perrito \textit{poodle} contribuyó mucho  al la formación como estilista de Tony. Con él ensayó diversos cortes; lo peor fué al principio, como no sabía, usaba rastrillo de doble filo, quedando la piel muy irritada, pero ``Pelusín'' siempre fué dócil.

\chapter{¡Adrenalina!}
cierto día llamaron por teléfono para solicitar la vacuna de tres perros pastor alemán, pero advirtieron que debía ser vacunados en su domicilio, porque eran difíciles de manejar.

Llegó mi esposo al domicilio indicado, y solamente habían dos señoras mayores, quienes aseguraron que podían con los perros. Indicándoles cómo ponerles el bozal, primero vacunó a los más bravos, dejando al último al mansito; al ir las señoras a poner el bozal al tercer perro, se les escapó, y cayó sobre mi gordo como una auténtica fiera. Mi esposo, al ir cayendo y perdiendo el conocimiento, alcanzó a patear instintivamente al perro, presumiblemente en la nariz, ya que éste salió corriendo ---Puesto que, con el coraje del animal, de haber sido golpeado en otro lado ni lo hubiera sentido---. 

 Al levantarse, mi esposo notó que tenía desprendida la aleta izquierda, le faltaba un diente incisivo, y tenía una dentellada profunda en la parte inferior de la barbilla. Las señoras corrían de un lado a otro presas del pánico, y sólo acertaban a decir ---``¡Ay doctor, ay doctor!''.

Iba con él un compañero que tenía coche, y que quería aprender lo que Tony ya hacía. Gracias a este amigo que lo llevó de inmediato a la Cruz Roja, pasando antes a la casa, posiblemente por dinero, no lo recuerdo bien, solo se que quedé impactada al ver a mi esposo con la mano sobre el rostro, escurriendo la sangre hasta el codo, y la ropa ensangrentada y llena de tierra.
No sabía cómo esperar el momento de volver a verlo y saber cómo estaba y qué había pasado.

Gracias al excelente trabajo que ahí hicieron, ---ni siquiera le quedó cicatriz--- y también gracias a una amiga que estudiaba odontología en la misma UNAM, y que le puso su incisivo con una técnica vanguardista para el momento, quedó el diente fijo con un tornillo, de manera que el rostro de Tony no acusó ningún  cambio, pero el incidente le dio mucho aprendizaje para tomar todas las precauciones posibles en el trato con sus pacientes peligrosos
\chapter{¡Gracias, Alma Mater!}
A mi esposo le tocó estrenar el edificio de la Facultad de Medicina Veterinaria y Zootecnia. 

Estaba tan nuevo, que diario encontraba fauna diferente, como: Alicantes, murciélagos, tlacuaches, zorrillos, liebres, arañas de varias especies, lagartijas y mariposas diversas, y un día se encontraron con un halcón.

Este edificio estaba enclavado en una zona rocosa volcánica, y mi Gordo cada día entraba emocionado a las aulas, para ver qué animales encontraba;Se encontró incluso un coyote.

En este lugar había mucha vegetación, se sentía fresco, bonito, se escuchaban pájaros diferentes; prácticamente este edificio ocupaba el último rincón de la universidad.
Allá al fondo se podía ver el Jardín Botánico de la UNAM, y más lejos, el volcán Xitle, que dio origen al Pedregal de San Ángel, y a toda esta zona.

También se podía apreciar el camino que iba al citado Jardín Botánico.

Resultaba muy agradable tomar las clases con esta panorámica de manera que en el recuerdo se fusionan el paisaje y los maestros.

El maestro más querido para mi esposo es el dr Alfredo Cortez Arcos, que tanto lo ayudó; él impartía la cátedra de ``Técnicas Quirúrgicas''.

Este doctor un día le dio un  consejo que hasta la fecha tiene siempre presente: Tomar toda clase de precauciones con las uñas de las patas de los felinos; se refería exactamente a las traseras, ya que él mismo había estado en terapia intensiva por esta causa. Le explicó que dichas uñas están más cóncavas que las de adelante, y albergan muchos gérmenes patógenos muy peligrosos.

También apreciaba al doctor Rodolfo Cuéllar, que impartía Anatomía Comparada.

Otro maestro muy estimado fué el dr. Alejandro Alejo, que tenía a su cargo la materia Anatomía Descriptiva.

También recuerda al dr. ``Cabrerita'', director del Zoológico, que en una ocasión citó a sus alumnos en éste, para realizar una operación estética a un tigre de Bengala, que había perdido un colmillo, al cual le implantaron uno nuevo. La cercanía con este espléndido animal, quedó grabado en su mente de modo
indeleble.

El  citado doctor opinaba que era muy importante realizar este trabajo, porque en un animal de esta clase, no era digno estar sin un colmillo, pues causaría lástima.

El dr. Berruecos dejó su imagen grabada en mi Gordo por su férrea voluntad; este médico hizo la maestría y el doctorado en Inglaterra, y venía determinado a la producción de vacas pequeñitas, pero productoras de leche. 

Su idea era que en poco espacio se pudiera tener un animal de estos, y que no faltara la leche; le preocupaban las familias de escasos recursos; era muy importante para él poner su granito de arena para tratar de mejorar la alimentación.

Este doctor exigía a sus alumnos que fueran cultos; porque era frecuente escuchar comentarios como que: ``Un veterinario es un médico de segunda''.
El mencionado doctor, al aplicar sus exámenes, obligaba a contestar el 30 por ciento de preguntas sobre su clase, que era genética, y setenta, de música clásica y pintura.

Tony nunca tuvo problema con la música, porque desde niño ha sido melómano. 

También recuerdo con cierta diversión que me hablaba del dr. Viterbo Cortez Lobato, porque su clase era a las 7:00 a.m.\ y con mucha frecuencia llegaba con saco, pero debajo llevaba la pijama, la cual podían  ver por los puños y el cuello salidos; completaba su atuendo con unas botas llenas de estiércol.

Este pintoresco maestro llegaba 5 o 10 mins.\ tarde, pero en cuanto él llegaba ya nadie podía entrar. Un día se tardó 15 mins.\ y decidieron sus alumnos no dejarlo entrar. 

Dicho doctor impartía la materia de IPOA---Inspección de Productos de Origen Animal---.
La dra. Angelita, por desgracia no le dejó mucha huella, sólo recuerdo que decía que era rubia y muy atractiva. 

Por último mencionaré a ``Pitirijas'', que no era del personal docente, pero tenía relación con la mayoría, ya que atendía el ``Departamento de Nutrición y Patología'', llamado así porque vendía tortas y les boleaba el calzado.

Un día le preguntó a mi Gordo dónde había comprado sus zapatos; Tony respondió muy orgulloso ``¡Son de León!!'', a lo que él le respondió:
``Con razón rugen tanto\ldots''.  A Tony siempre le llamó ``Mi León'' y seguido le preguntaba por la familia Sepúlveda (Conocida familia de veterinarios leoneses).

En estos momentos ya comenzaba a despedirse de la Facultad y todo lo que incluía: Maestros, compañeros, vivencias, rutinas, etc.

\chapter{Criadero de Perros ¡Rumbo al éxito económico!}
Un amigo --- que posteriormente sería sacerdote --- nos presentó a su familia, y el señor Rosales --- que era el papá --- de inmediato puso sus perros a su cuidado; vivía cerca de nosotros, y además, era leonés. Con el tiempo surgió una amistad firme, y un día en  que platicaban, el señor dijo: ---`` ¿ Por qué no ponemos un criadero de perros?. Tú tienes lugar, y a mí me queda dinero libre de mis camiones urbanos.''---.

Ciertamente contábamos con espacio, porque el terreno de al lado también pertenecía a la familia, y había una puerta entre la cocina y el baño, que comunicaba hacia esta área.

Salvo la parte de adelante, que ocupaba un taller mecánico, el resto era terreno baldío. Un día ---casi recién casados--- reuní energías, y me decidí a limpiar este terreno, que estaba abandonado y con hierba muy crecida; no recuerdo cuanto tardé, pero quedó limpio, y mi Gordo colocó tendederos de lado a lado. Ahora sí tenía espacio apropiado, y éste no estaba a la vista. Me sentí muy contenta con esta nueva área, ya que me avergonzaba tener ropa secándose al frente de nuestra casa, ademas de lo feo e incómodo de estar pasando entre ropa tendida.

En esos días supimos que en algún lugar, donaban arbolitos, y fuimos por algunos, los cuales plantamos y protegimos para que crecieran bien. Luego Tony hizo un sendero que cruzaba el patio de lado a lado, bordeado por lirios blancos y morados. Pusimos alguna banca por aquí, otra por allá, y ya teníamos nuestro jardín, sin ningún peligro para los niños.

Fue después de esto cuando los dos amigos hablaron sobre poner el criadero, lo cual concretaron afirmativamente.

Como inicio el sr. Rosales entregó 5000 pesos. Esta cantidad quedó registrada en una libreta comprada ``ex profeso''--- ya que sería todo ``limpio y transparente''--- en la que podía leerse ``Criadero Rosanda'' (de Rosales y Aranda)---.

Tony compró malla ciclónica, para hacer una gran perrera, con compartimentos para cada animal. Decidieron que para evitar accidentes, sólo habría hembras de la mejor calidad.

Recuerdo muy bien que la primera hembrita que compró era una \textit{Chihuahua} blanca, muy chiquita y bonita, que, mientras, vivía con nosotros. En esos días tuvimos que venir a León, y nos trajimos a la perrita, ya que no podía quedarse sola. Esta era la primera salida a carretera, con nuestro primer carrito, y de paso pedimos al párroco de ``La Conchita'', ---Nuestra Sra.\ de la Concepción Tequipehuca, la parroquia cercana a nuestra casa, que ostenta una placa que dice que ahí fué capturado Cuahtémoc--- que lo bendijera, y lo hizo, pero diciendo que solo tenía garantía si no se excedían los 100 km/h. Al pasar por Querétaro, pusimos gasolina, y ¿en cuál de las dos paradas se saldría la perrita?, no lo supimos, pero no llegó a León. 

Esta fué la primera pérdida para el negocio.

Tiempo después, mi Gordo compró una \textit{cocker spaniel} de color muy claro, llamada ``Wendy'', una \textit{boxer}, una \textit{cocker} negra, una pinta, y una caoba. Por más cuidados y precauciones que tomamos, ``Wendy'' se cruzó con ``Pelusín'', que era \textit{poodle}; tuvo cuatro cachorros. Esto hubiera sido solo un tropiezo, si no se hubiera salido la perrita, y no se hubiera metido a una vivienda vecina, y no se hubiera comido el sebo con veneno para ratas, pero fué una gran pérdida, porque la hallamos muerta. 

Tuvimos que terminar la crianza de los cachorros, y acomodarlos, lo cual no fué muy fácil, ya que no era ni de una ni de otra raza.

Este negocio, en lugar de aumentar, decrecía.

Por otro lado, ya iba siendo tiempo de volver a León. Nuestro convenio era que viviríamos en México mientras se concluían los estudios, y ya habíamos completado 7 años.

Mi esposo ya había elegido su especialidad, que era pequeñas especies, y calculaba que en León las personas no atendían igual a sus mascotas, y quien sabe si tendríamos ingresos.

Pero lo que nos hizo decidir sin titubear, fué que nuestro niño mayor terminaba el kinder y tendría que entrar a la primaria, y yo escuchaba cosas horrorosas, (Como que en el baño los más grandes abusaban de los chiquitos), y yo todo lo que oía se lo comunicaba a Tony.

Al mismo tiempo, ---Bendita coincidencia---, mi mamá me habló por teléfono, preguntando si ya íbamos a regresar, porque estaba arreglando una casa, que si queríamos nos la reservaría; con todo esto, empecé a empacar.

Por su lado, mi Gordo habló con el sr. Rosales, explicando las pérdidas, y en general nuestra situación, prometiendo devolver el dinero en cuanto fuera posible, a lo cual contestó el sr. ---`` que conocía los riesgos, que había  sido su decisión, y que además no siempre se gana''---.

Y así terminó la historia de este criadero.

Como ya dije, nos dispusimos para regresar. Una amiga y comadre me enseñó a empacar, y lo hizo tan bien que nada se rompió. Ellos también ya pensaban regresar, pero no a León; eligieron Celaya, y desde entonces mi compadre ha desempeñado con éxito la carrera de otorrinolaringología, siguiendo mi ahijado los pasos de su padre.

Nuestra casa, donde vivimos nuestros primeros años de matrimonio, a veces felices, a veces con muchas privaciones, se la dejaríamos a un buen amigo y su familia, a quienes prodigábamos verdadero afecto. También se encargaría de recibir las rentas, de las tres viviendas y el taller mecánico.---Para esta fecha ya había muerto la tía de Tony, que ocupaba el piso de enmedio.---.

También muy cerca había una vecindad que formaba parte de la misma propiedad, pero que tenía rentas congeladas; aún así había que recoger el dinero, y nuestro amigo lo haría; este poco dinero pertenecía a mi suegra.

Fuimos diciendo adiós a todo: Amigos, vecinos, lugares, peligros, escasez de agua, aprendizaje y experiencia\ldots
\chapter{El regreso}
Era la mañana del 31 de agosto de 1977, cuando finalmente subimos a nuestro coche, de modelo más reciente que el anterior, y más grande: se trataba de un Opel Olímpico blanco.

Acomodamos a nuestros tres hijos, un niño y dos niñas, y finalmente salimos de México. Yo constantemente volteaba hacia atrás, complaciéndome en la comodidad con que viajaban los niños, con sus propias almohadas, y recordando años anteriores, en que lo hacíamos en autobús. Me deleitaba con el pensamiento de que un auto propio es la continuación de nuestro hogar.

Llegamos por la tarde, y dormimos en la casa de mi mamá, y al día siguiente, temprano, nos fuimos a nuestra casa vacía para esperar la mudanza.

Esta casa era pequeña, pero suficiente para 5 personas, todavía faltaban dos años para que tuviéramos a nuestro cuarto y último hijo. Nuestro hogar estaba ubicado a igual distancia, en un punto intermedio, entre la casa paterna mía y la de mi esposo.

La mudanza llegó esa misma mañana, y comenzamos a instalar lo más urgente, pero a mi Gordo lo veía inquieto, y es que deseaba salir a buscar un lugar para su trabajo. 

Mi mamá le informó que frente a su casa, estaba una vacía, y la tenían en renta; el fué a verla, y no le gustó, dijo que estaba enorme y deprimente.

Diariamente salía a buscar, y guiado por el periódico, visitó muchas, pero ninguna le parecía adecuada, o bien estaban muy caras. Así se pasó un mes.

Por mi parte, yo estaba exultante, por tomar posesión nuevamente de mi ciudad, caminar conscientemente por cada calle, como lo hacía. Me proporcionaba enorme placer reconocer los lugares de mi niñez, despertaban en mí muchas emociones. 

Después de un mes de buscar, Tony tuvo que volver a la primera que había visto, y por suerte, estaba todavía sola; su precio era accesible, y viéndola bien, solo era cuestión de cambiar el color de la pintura, de modo que se arregló con la dueña, que vivía a un lado, y comenzó las reparaciones. Solamente utilizaría la entrada, el resto quedaría desocupado, pero con el tiempo, les servirían tantos cuartos para aislar perros enfermos e infecciosos. 

Yo tuve que volver a aprender a manejar, ya que en México nunca me atreví, y aquí debía llevar y traer a los niños a la escuela.

Mi Gordo comenzaba a desesperarse: El poco dinero ya se había ido en arreglos y renta, y además se sentía inseguro, pensando que tal vez no tendría trabajo. Finalmente vendió su bicicleta de carreras en 5000 pesos, y con eso se fué a México, y regresó con cadenas para perro, pecheras, peces, accesorios de varios tipos.

Acondicionó una pecera que había hecho cuando era adolescente, y en un librero exhibió la mercancía, que se vendió toda, y en una semana ya tenía el doble del dinero inicial. Se fué otra vez a México, y así continuó. 

Después de tiempo ya eran 40 peceras y el lugar lleno de clientes y de mercancía; el coche lo tuvimos que cambiar por una camioneta, con \textit{camper} para traer peceras, bultos de alimento para diferentes animales, etc. 

Ya después, mi esposo era muy conocido en ``El corralón'' donde iba a surtirse, y los dueños de un puesto grande y bien surtido, le ofrecieron mandarle todo a León con solo hablar por teléfono. De este modo el trabajo se volvió más cómodo. Al mismo tiempo iba equipando su consultorio; su fama corrió y sus temores quedaron muy atrás.

Nuestros hijos fueron creciendo entre muchos animales, y todos ayudábamos a atender al público, sobre todo, si el veterinario, un poco más adentro, en su consultorio, estaba atendiendo alguna mascota.

Entre tantos animales, un día llegó un pelícano, que una persona había encargado. Era pollo, le faltaban plumas, pero ya tenía el tamaño de un adulto.
 La persona tardó unos días en recogerlo, mientras tanto era la atracción, y nosotros estábamos asombrados de lo mucho que comía; le dábamos más pescado que el que consumía toda la familia, y aún así seguía pidiendo, Ja, ja, ja hoy sabemos que consumen mucho mas.
 
 Después que se llevaron al pelícano, --- Uf, era mucho gasto alimentarlo--- trajeron a consulta un venadito igual a Bambi, con sus manchitas igual, moviendo su colita graciosamente. Mientras se aliviaba lo dejaron hospitalizado, nosotros lo sacábamos para que caminara, y no nos imaginábamos cuanta promoción nos daban criaturitas como estas.
\chapter{¡Peligro!}
Un día le pidieron a mi Gordo un cachorro de león, el cual pidió de inmediato a México. Fuimos por él al aeropuerto, y era mucho el asombro que causaba su presencia.

El futuro dueño solicitó que al animal le fueran extirpadas las garras, y hasta entonces se lo llevaría. Tony se dio cuenta de que la anestesia que tenía no servía para este animal, y mientras indagó, pidió y le mandaron la indicada, pasaron como dos meses. El animalito apenas tenía 3 meses, y no mostraba señales de agresividad, solo trataba de jugar como cualquier cría.

Lo sacábamos a pasear con cadena, para que se ejercitara. En una ocasión lo traía yo, y lo jalé para entrar en la casa, y me vi en apuros, porque se resistió y gruñó. Esto me hizo ver que no se trataba de un perro. Ya dentro de la casa, le quité la cadena, y al pasar él junto a una silla forrada de piel, bostezó y se estiró sobre la silla, la cual quedó rasgada como si cada uña fuera una navaja afilada.

Con los días se iba haciendo cada vez más difícil su manejo, de modo que debíamos cerrar puertas y mantenerlo alejado. Para poder entrar al área reservada al animal, que era la mayor parte de la casa, teníamos que usar escobas para obligarlo a retroceder, y solo así podíamos alimentarlo y limpiar el área donde habitaba.

Las escobas siempre estaban a la mano, ya que eran cada vez más frecuentes sus intentos de trasponer la última puerta que lo confinaba en su morada. Esta puerta tenía su mitad superior de cristales, y faltando un cristal, el felino parado sobre sus patas traseras, podía alcanzar fácilmente el hueco y saltar a través de él. Con las escobas lo ahuyentábamos sin lastimarlo, si teníamos suerte antes de que lograra saltar. Por su parte, el ya había tomado posesión de toda la casa\ldots

Un día, no supimos cómo, nuestro hijo menor, que tenía tres años, se metió al área del león. Al darnos cuenta de que no estaba con nosotros, corrimos hacia adentro buscándolo; lo encontramos agazapado en un rincón, queriendo protegerse, porque el león lo tenía acorralado; sólo al vernos, pidió ayuda aterrorizado. Con las escobas y una cuerda, alejamos al animal y pudimos sacar al niño. Todos estábamos temblorosos\ldots

Pocas semanas después, este mismo niño, se acercó gateando, y con toda confianza, a un ``antiguo pastor inglés'' que llevaron a consulta; este animalote volteó agresivo, golpeando con los dientes la cabeza del niño. Al verlo sangrar muy cerca del ojo, lo llevamos de inmediato, con el médico, y gracias a Dios, resultó sólo una herida pequeña. Fué entonces cuando nos pusimos a analizar, y nos percatamos del gran peligro en que había estado el niño, porque el perro bien pudo haberle roto el cráneo, ya que la cabeza del niño cabía perfectamente en la boca del animal.

Otro día, mi Gordo llegó con un \textit{kinkajú}, o \textit{martucha}, muy mansito, ¡Era para nuestro hijo el mayor, el cuál se alegró mucho!. Muy pronto se hicieron grandes amigos.

Lo traía siempre en el hombro, y el animalito subía y bajaba con toda confianza. Cuando ya se habían acostumbrado uno al otro, un día de pronto, al llevarlo a su lado con su cadena, sin previo aviso ni motivo aparente alguno, el volteó con una mirada llena de hostilidad, y comenzó a escalar la pierna izquierda, hasta hincar los colmillos en el borde inferior izquierdo de la cara, mientras usaba ambas zarpas para dejar en mal estado todo el costado izquierdo de la cara de mi hijo. El muchacho lloró de susto y desilusión; lastimado y ofendido, ya no le hizo caso al animal, y días después, había desaparecido.

A la semana siguiente, llegó un joven con el \textit{kinkajú}, quejándose de que le había roto las vestiduras del coche. Al ser interrogado sobre la forma en que lo había obtenido, contestó que vivía en el edificio de la esquina, y que una noche había entrado por la ventana a su habitación; y como era mansito, no indagó de donde era. Hasta que lo perjudicó. Entonces sí supo a donde ir\ldots

\chapter{Nuestra casa grande}
Oigo repiquetear las campanas del templo que está a unos pasos; se debe a que es el día de Nuestra Señora del Carmen. Nos estamos cambiando de casa, nos vamos a donde está la clínica. Con la ayuda de unos amigos de nuestros hijos lo vamos a hacer en un solo día. Nuestros hijos ya son adolescentes, solo el menor es un niño de 10 años. 

Hemos acomodado lo más indispensable trabajando todos arduamente, pero llegó la noche y las recámaras de atrás todavía no tienen luz, y les he dado una vela a cada uno, pero Damián, el menor, está llorando porque tiene miedo.

Hemos pasado la primera noche aquí, y al despertar descubrí la gran diferencia, solo se escuchan pájaros en medio de un gran silencio. ¡Que agradable!

Me he levantado de prisa, porque hay mucho quehacer y además vendrán a comer unas personas. Tengo la sensación de estar en una hacienda y esto es justamente lo que deseábamos. Pensé que sobraría lugar, y no es así; simplemente estamos más amplios, y hacen falta muchas plantas, macetas y faroles. ¡Estoy tan feliz!. Este cambio ha sido muy importante para mí, significa un gran regalo que deseaba hace mucho tiempo y ha resultado a la medida de nuestro gusto.

\chapter{Rarezas}
Un día, estando mi Gordo analizando el agua del las peceras, sonó el teléfono, le pedían que fuera a domicilio para ayudar a una perra Gran Danes, que estaba postrada, sin poder parir. Fué enseguida, a pesar de que siempre ha preferido trabajar en su clínica, por la altura de las mesas, instrumental y medicamentos a la mano.

Al llegar, toda la familia rodeaba a la perra. A duras penas y empujando se hizo el espacio suficiente para comenzar su labor.

Empezó por aplicarle una inyección, aunque las cabezas de todos los miembros de la familia, le impedían la visibilidad, y enrarecían la atmósfera. Sentía a uno en un hombro, otro en el otro hombro, otros más arriba, otros por el frente, todos queriendo ver.

Por fin apareció la primera colita, y sólo tuvo que jalar al llegar la contracción, y ya estaba afuera, rompiendo rápidamente la placenta, se lo dio a uno de tantos ayudantes, para que lo masajeara hasta verlo respirar, habiendo ya limpiado las vías respiratorias, cuando ya venía el segundo, repitiendo el procedimiento y pasándolo a otro ayudante. Y luego el tercero, y así continuó hasta que quedó vacío el útero, contando 21 cachorritos, 5 horas de trabajo, y sintiéndose muy agradecido por contar con tantos ayudantes. De 21 cachorros sobrevivieron 19, y la madre, luego de 3 días de trabajo de parto, murió al siguiente, teniendo que hacerse cargo la familia de los cachorros. A unos les tocaron 3, a otros 2, y lograron sacarlos adelante.

En otra ocasión recibió en la clínica a una perra San Bernardo, preñada y en muy malas condiciones. Después de auscultar e indagar el tiempo de gestación, supo que tenía que realizar una operación cesárea.

Al abrir, --- ¡sorpresa!--- se encontró con dos úteros. Abrió el primero, y vio que todos estaban muertos; al instante, se llenó el ambiente de un olor nauseabundo; limpió el útero muy bien, y abrió el segundo, y también estaban muertos, pero muy sorprendido comprobó que la gestación no había empezado al mismo tiempo en ambos úteros. Hace ya muchos años de esto, y nunca se le ha vuelto a presentar un caso similar.

Otro día le llamaron para revisar un perro al que no podían convencer de salir a la calle. Cuando llegó se quedó sorprendido de la fealdad del animal: Tenía la cabeza grande, de \textit{bulldog}, y un cuerpo chico, alargado, con extremidades cortas, de \textit{dashound} ---`` salchicha''---. Lo atendió, pero salió muy impresionado con este capricho de la naturaleza.

\chapter{Mitos}
Recuerdo al doctor hablando con sus clientes, y luchando contra muchos mitos. Con frecuencia le pedían que curara al perro, que le daban ataques, y era porque ``había comido pan con agua'' y esta mezcla originaba lombrices. Estas eran la causa de los ataques.

El les explicaba que era conveniente desparasitar al animal cada determinado tiempo, pero que también había que considerar que la harina del pan la blanqueaban con una sustancia que contenía potasio, y algunos perros eran sensibles a este, y podían presentar convulsiones, pero que no era cierto que el pan y el agua originaran las lombrices.

Está muy arraigada la creencia de que ``matriz que no da hijos da tumores; es muy frecuente que las personas, llenas de aprehensión, pregunten sobre este tema, el doctor contesta lo mejor posible, tratando de dar tranquilidad. 

También se divulgaba y se divulga la creencia de que los mejores cachorros se obtienen de padres que estén emparentados muy de cerca\ldots¿Y la genética?\ldots

\chapter{Animales Traviesos}
En nuestra tienda de animales ha habido variadas especies, pero lo que yo nunca he podido tocar son las víboras y las tarántulas; son inofensivas, pero no puedo dominar mi repulsión.

Si me pedían uno de estos animalitos que tanto atraían a niños y jóvenes, y no había nadie más que atendiera, yo les decía:---`` Escoge el que más te guste''--- y ellos contestaban: ---``No, mejor usted deme ese''--- y yo respondía:---``Si te lo vas a llevar, más vale que empieces a manejarlo''--- librándome así de tocarlos.


Sucedió que un día mi hijo traía en la mano una tarántula \textit{brachiophelma}, que alguien deseaba comprar, pero temían que fuera peligrosa. Él la pasaba de una mano a otra, y pidiendo mi apoyo, me preguntaba, ---¿Verdad que no hacen nada?---; yo me encontraba de espaldas limpiando la estantería y afirmando que estos arácnidos eran inofensivos, no me dí cuenta que de su mano brincó a mi cabello, y subió hasta lo alto de mi cabeza. Mi hijo ya se estaba tapando los oídos, porque sabía lo que seguía: Unos gritos que se deben haber oído a una gran distancia. Cuando mi hijo recuperó la tarántula, los clientes incrédulos me preguntaban: ---¿Si no hace nada por que grita tanto?---.

Estas tarántulas, sin ser agresivas para el humano, si impresionan, pues tienen el tamaño de una mano grande, son negras con rayas amarillas en las patas. Cuando se sienten amenazadas desprenden unos pelillos urticantes.

Nuestra vecina, que primero nos rentó y luego nos vendió la casa, había dividido esta casi a la entrada con un bastidor de madera, quedando muchos resquicios. Así dividida la  casa, ella se quedaba con la parte chica, y nosotros con la grande.

Yo calculo que nuestro hijo más chico anduvo jugando, abriendo las puertas de las jaulas, y seguramente dejó abiertas algunas. Cuando regresamos al día siguiente, habían desaparecido todas las víboras, las tarántulas y dos ardillas.Apenas habíamos entrado, cuando sonó el teléfono, era la señora de al otro lado, muerta de miedo, que nos pedía que sacáramos todos esos animales de su casa, que ya había telefoneado a su esposo para que viniera de su trabajo a abrirnos; ella no podía hacerlo, porque estaba sentada en lo alto de la cabecera de su cama. Al llegar el marido, y verla tan asustada, nos dijo que si volvía a suceder, nos demandaría. Al parecer fueron atrapados todos los animales, pero las ardillas no aparecían. La señora ya estaba muy tranquila, platicando amenamente  por teléfono con una amiga. Cuando descubrimos nosotros que por la hendidura que tenía el cancel de madera se habían pasado las ardillas, y podíamos ver su cola, de inmediato fué mi Gordo con una red para atraparlas, y antes de tocar, pudo ver a la señora hablando por teléfono y descubriendo en ese momento también a las ardillas, y gritándole a su amiga ---``¡¡Esos animales me están viendo!!''--- y entre grito y grito, se lanzó una de las ardillas directo a su cara, la señora gritaba y corría y mi Gordo impotente, viéndolo todo, pero con la puerta cerrada. A través de la ventana le rogaba el que se tranquilizara y que abriera. Cuando al fin pudo hacerlo, Tony atrapó a las dos ardillas, y no hallaba palabras para disculparse. De verdad que nos apenó mucho, y sentíamos mucha vergüenza, de haber causado tanto tormento a la pobre señora. Pero también  es cierto que nos reímos mucho, y aún seguimos haciéndolo.

Cierto día apareció en la clínica un tlacuache, claro que en plena zona céntrica, esto no es posible, alguien lo llevó, y por \textit{x } causa se quedó. Lo vimos muy tranquilo, y había estado así todo el día, de modo que, confiados, nos fuimos a nuestra casa a dormir. A llegar a trabajar al día siguiente, en cuanto abrimos la puerta, nos quedamos estupefactos, sin entender la hecatombe que veíamos.

Toda la mercancía de las estanterías había rodado por el piso. Con gran trabajo y lentitud fuimos acercándonos, y vimos al culpable de todo: El tlacuache. Nos quedamos asombrados al ver que no era macho, sino hembra, pudimos ver el marsupio, (bolsa) de la que se asomaban cinco cabecitas asombradas.

Fue una proeza atraparla, meterla a una jaula, para después llevarla al campo y soltarla.

En otra ocasión fué un zorrillo el que llegó y se quedó igual que el animal anterior; despues de recorrer toda la casa, se le ocurrió meterse en el closet de blancos que teníamos en el baño; yo alarmada, pedí a mi hijo que sacara de ahí al animal, peero sin que arrojara nada pestilente. En eso vimos como caminaba por el cielo raso, dejando perfectamente claro su rastro. A duras penas el muchacho logró atraerlo a la salida, y cuando ya no había peligro con la ropa, lo bañó a el completamente, y el olor llegaba hasta media cuadra.

Mi hijo se bañaba y restregaba, pero no lograba hacer desaparecer el olor; todavía no sabíamos que para estos casos es muy útil la salsa catsup.Igual que con los otros animales, tuvimos que llevarlo al campo, pero muy bien tapado y evitando que se asustara y soltara su líquido. 

Otro día el que llegó fué un mapache; era muy divertido verlo comer, porque estos animalitos acostumbran lavar su comida, la restregan y enjuagan varias veces usando sus dos manitas. A este animalito lo dejábamos libre por las noches. Por ese entonce teníamos a la venta cachorros de antiguo pastor ingles, y para cuando llegó el mapache, ya solo quedaba uno.

Una noche oímos que gritaba el perrito, pero dimos por hecho que jugaban, hasta el día siguiente nos dimos cuenta que le faltaba un dedo de una patita. El mapache se lo había arrancado y lo tuvimos que llevar a su lugar de origen.

Con el cachorro no hubo problema, porque al cliente que lo compró le gustó tanto el perrito que no le importó.

Cierta vez, Tony compró un mono araña, y al ir a nuestra casa a comer, nos lo llevábamos por la calle con una cadena. El usaba esta cadena como columpio, íbamos y veníamos con el por diversión, y también para promocionarlo. Mi Gordo notó que tenía una cicatriz en la barbilla, y al verlo cómo reaccionaba ante una escoba, comprendimos que había sido maltratado.

Uno de esos días, regresábamos de comer, trayendo al monito, y ya casi para llegar, salió una vecina vestida de rojo y muy divertida observaba al mono, cuando lo acarició el changuillo se alteró tanto como no lo habíamos visto, se le enredó con la cola, las manos, las patas, mezclándose los gritos desesperados de ambos. Al tratar de quitárselo, con la cola le levantó el vestido; logramos desprender la cola del vestido, pero con las manos jalaba el escote. La pobre vecina no hallaba que taparse, jalaba el vestido para arriba y luego para abajo.Después de unos 15 minutos de lucha logramos sujetar las extremidades, pero hasta del cabello la jaló, mandándola asustada y desgreñada, y desde luego, nosotros también quedamos asustados y agitados, pero (la verdad), a duras penas logramos contenernos de soltar la carcajada.

Lo curioso es que a pesar de esta experiencia, tiempo después ella misma nos encargó otro igual, pero más chiquito; lo quería como atracción para su negocio (zapatería).

Cuando llegó por fin este monito, mientras lo recogían estaba en su jaula, jugando y comiendo muy tranquilo; en eso llegó Benjamín, un empleado de mi mamá, que por lo tanto, trabajaba enfrente. Este hombre de cabellera abundante y crespa, que al parecer no estaba enterado de la invención del peine, llegó para que le cambiáramos un billete. Le dijimos: --- ``Mire que bonito changuito\ldots''--- señalándole la jaula, pero el se acercó tanto que el changuito gritó, y fué tal su susto que se refugió en el fondo de la jaula, chillando desaforadamente\ldots El pobre de Benjamín, intrigado, preguntaba ---``¿Por que grita tanto?''---, nosotros sólo nos reíamos, pero no le pudimos decir que era a causa de su fealdad. Esta vez, fué al revés.

Cuando por fin vino su dueña por el, y se lo llevó a su local, los niños que salían de la escuela y se detenían a verlo, estaban encantados con el monito. La dueña de la zapatería ofreció un premio para el que propusiera el mejor nombre, y fijó un día para este fin.
Este día en que se llevó a cabo el concurso se llenó la zapatería, ganando el niño que propuso que el changuito se llamara ``King-King''.  Este monito no vivió mucho tiempo, precisamente por tantos niños, que le daban todo tipo de golosinas; al no tener una alimentación adecuada, se fué debilitando poco a poco, hasta que enfermó y murió sin recibir atención médica.

Un tiempo después compró Tony un mono \textit{zarahuato}, que se portaba muy tranquilo. Llegó un taxista, a quien le gustó, y lo compró, ya que quería ser la atracción entre los de su gremio. No pasó una semana, cuando regresó todo batido de diarrea de mono; se presemtaba a  regresar al chango, diciendo que así como estaba él, así le había dejado el taxi, y no quería más de estos suscesos, aunque no se distinguiera de los demás compañeros.
\chapter{Inteligencia animal}
La clínica veterinaria de mi Gordo no estaba en un local, sino en una casa grande de la zona centro, así que con la clínica, y tantos animales, sólo se ocupaba la mitad de esta casa, y ya empezábamos a cansarnos de ir a nuestra casa a comer, y regresar de prisa por que ya nos estaban esperando.
De manera que después de pensarlo, decidimos hacer algunas reparaciones e irnos a vivir a la casa del centro, extendiéndonos para ocupar la parte mayor restante.

Nos sentíamos muy holgados en esta casa tan grande, y pensamos  que ahora sí podíamos tener un perro grande, como todos queríamos.

Al poco tiempo, al estar atendiendo el doctor a una mascota, alguien llegó con una perrita de cuatro meses, hermosa, de raza \textit{pastor alemán}. A simple vista el doctor apreció la calidad, pero lo que le encantó fué que en vez del dorado, tenía un vivo color cobre, que con el negro del dorso resaltaba bellamente.

Por estar ocupado, Tony solamente la revisó de un modo rápido, pagándola y pidiendo que la metieran en la jaula de abajo.

Después de mucho rato pudo entrar a decirnos que ya teníamos nuestra perra, todos corrimos a conocerla, y la integramos fácilmente a la familia, observando de inmediato su gran inteligencia. Entre los tres hombres, la educaron para andar en la calle; rápidamente aprendió a caminar al paso de su conductor, siempre de lado izquierdo. Obedecía muy bien a la palabra ``SIT''; ya después ella sola se detenía y se sentaba al llegar a cada esquina, hasta recibir la orden de continuar. No debía correr, ni atravesarse al paso del conductor, ni retrasarse, y todo esto lo aprendió muy pronto. 

De carácter muy sociable, recibía a las visitas, y también las salía a despedir hasta la puerta de la calle, pero esto mismo hacía con cada uno de los perros que llegaban a consulta, infecciosos o no; debíamos estar muy al pendiente, para ordenarle que entrara a su jaula, ``la de abajo''. También debimos tener mucho cuidado con sus vacunas y desparasitaciones, pero aún así, se contagió de \textit{moquillo} (enfermedad muy grave entre los perros).

Estando en tratamiento, y postrada en su cama, una noche se levantó a rasguñar la puerta de nuestra recámara, al salir mi Gordo, se dio cuenta de que no podía respirar, y  la tuvo en cuidados intensivos, quedándose con ella el resto de la noche.

Tuvo una rápida recuperación, pero le quedaron secuelas, como pasa casi siempre en los sobrevivientes a esta enfermedad; nuestra perra quedó afectada del corazón, pero aún así se desarrolló bien, y siempre fué alegre y participativa. En días de campo, o campamentos la perra siempre iba con nosotros, y se divertía acompañándonos en nuestros largos paseos por el campo.

Su inteligencia le dictaba cosas que nadie le enseñó: Dejaba entrar a todas las personas, pero si querían tocar algo de las estanterías, ella se interponía, detrás de los mostradores, parada y ladrando. Ante este guardián formidable nadie se atrevía.
Recuerdo muy bien como la menor de nuestras hijas y esta perra se comunicaban muy bien: Por ejemplo, me causaba mucha gracia que mi le preguntara;---``¿Quieres un plátano?''---, y la perra volteaba hacia donde estaba la fruta, y comenzaba a salivar hasta escurrirle\ldots

Con el menor de mis hijos aprendió que la mano derecha se llamaba ``la mano'', y la izquierda se llamaba ``la otra''. 

Pero la verdad es que con todos se comunicaba bien, a cada uno nos trataba de acuerdo a como éramos con ella, con mi hijo mayor podía manotearlo, y al menor hasta le daba pequeñas mordidas jugando a las luchas; con las mujeres era más delicada.  

Si le decíamos ---``Ve con papá''---, iba hasta donde él se encontraba.

En tiempo de posadas se divertía mucho con el confetti, la piñata y los globos, atrapando cada cosa que caía, y al día siguiente el excremento era de muchos colores, lleno de globos. 

Pero no sólo esto: Como eran muchas personas, se sentaban en el patio y los corredores, dejando su bebida en el piso; la perra pasaba por detrás probando todos los vasos sin que se dieran cuenta sus dueños.Nosotros lo sabíamos al día siguiente, porque no se levantaba de su lugar favorito, y casi no abría los ojos aunque le habláramos; era tan obediente, que esto nos hacía sacar conclusiones.

Y de verdad que era obediente, cuando llegaba un perro grande y nervioso, que podía implicar peligro, Tony le ordenaba meterse a su jaula, al instante se metía, y además, jalaba la puerta de esta para cerrarla; obedecía reprimiendo su curiosidad, aunque ya estaba casi totalmente desarrollada y casi no cabía ya, ella seguía obedeciendo.

Recuerdo claramente que a la hora de comer, no tenía permitido entrar a la cocina, donde comíamos; se quedaba echada en la puerta, por fuera, pero sin dejar de observarnos. Cuando terminaba de comer nuestro hijo menor, comenzaba a jugar con ella, permitiéndole lo que el quisiera, y siempre terminaba practicando a andar en moto sobre la perra: Las orejas eran los manubrios, y cambiaba de velocidad, frenaba y aceleraba, imitando el ruido con la boca, y dándole vuelta a las orejas, y ella sólo cerraba los ojos y aguantaba.

Comía muy bien sus croquetas, pero también esperaba comida nuestra, y le dábamos los restos, pero nunca aceptó frijoles ni garbanzos, los dejaba muy limpios a un lado. También ignoraba la tortilla.

Estando la tienda cerrada, nadie podía entrar sin provocar su reacción. Recuerdo una ocasión en que la puerta estaba solo emparejada, y entró una pareja joven, y volvieron a cerrar; en cuanto la perra los vio se levantó, y a toda velocidad cargó contra ellos; el muchacho ganó la puerta antes, abrió, salió y volvió a cerrar, sin acordarse, en medio de su propio terror, de que su pareja estaba adentro tratando de abrir la puerta que él mantenía cerrada con todo su peso. Lo bueno es que la perra no era agresiva, solo celosa de su deber. Al llegar junto a la joven, esta solo cerró los ojos, esperando lo peor. La perra, al no sentir amenaza ni para ella ni para su familia, tan solo la olfateó brevemente.

Esta perra tan sólo estuvo con nosotros cuatro años. Me acompañó a la puerta de la calle, con intensión de acompañarme como siempre, pero yo le ordene volver; no podía llevarla conmigo; al llegar a mitad de la tienda, la vi caer pesadamente. El moquillo estaba cobrando su factura. Yo grité, mi Gordo llegó corriendo para tratar de reanimarla, pero era tarde. No había nada que hacer. Su corazón había cedido. Fué un gran duelo familiar. Todos la lloramos. Un miembro de la familia nos había dejado\ldots
El doctor sugirió la incineración, a lo que todos aceptamos. Junto con ella se fueron su cama, su plato, su trailla, sus cobijas, sus platos, ya que no queríamos que la vista de estos objetos hiriera más a la familia\ldots


\chapter{Solidaridad.}

Cierto día llegaron a la clínica unas personas trayendo un loro lastimado, el cual cayó en su jardín. El doctor lo revisó, y vio que tenía un ala dislocada, lo vendó, y para que no se quitara la venda, le puso cinta adhesiva, muy resistente. 

Al día siguiente regresaron las mismas personas, con el mismo loro, pero sin venda. Mi Gordo les preguntó qué había pasado, y le respondieron que entre le habían quitado la venda y la cinta; el doctor intrigado preguntó ---¿Como todos, quienes?---
¡Sus amigos!---¿Cuales?---¡Pues los de la parvada!---. Al entender la situación, pidió más detalles; ellos dijeron que estaban como 10 loros posados en la orilla de la azotea, habían bajado y lo habían liberado de la venda, y lo empujaban hacia arriba; él también lo intentaba, pero finalmente no lo consiguió. Tony, muy sorprendido por tanta solidaridad entre animales, les pidió a estas personas que lo mantuvieran informado, porque le interesaba mucho esta conducta. Volvió a vendar al loro, y esta vez le prometieron que lo encerrarían hasta que hubiera sanado, para luego liberarlo.

Pero ocurre que estas personas no han vuelto, así que no supimos mas\ldots

Hace muchos años, en esta ciudad comenzó a formase una parvada de \textit{psitácidos}, con todos los animales que se iban escapando, sabemos que así es, porque también a nosotros se nos escapó un loro ecuatoriano, de color gris azulado y cabeza blanca. Dos veces al día lo veíamos pasar con esta parvada, haciendo mucha alharaca. 

Actualmente han aumentado, se pueden contar como unas seis parvadas, y hemos comprobado que algunas especies han podido reproducirse aquí, sin ser su lugar de origen.

Platicando lo sucedido a otra persona, le contó al doctor que también había observado algo parecido: Le contó como un loro, al no poder volar, fué auxiliado por sus compañeros, quienes, pasando un ala por debajo de el, se lo habían llevado.Será verdad o no, no lo sabemos.

El único paciente que ha insultado al doctor, fué un perico que trajeron por que no quería comer. El médico, para poder revisarlo, le aprisionó la cabeza, evitando así ser mordido; lo auscultó, le tomó la temperatura y comprobó sus signos vitales, y por fin lo soltó. El loro se fué caminando a lo largo de la mesa, alejándose de mi Gordo y llamándolo ``Burro, burro''.


\chapter{Ternura}
Una noche en que ya nos encontrábamos durmiendo, sonó el teléfono. Era una voz amiga, que pedía ayuda. Mi Gordo percibió su aprehensión, y accedió de inmediato a atenderla. 

Llegó con un pollito que había sido atacado por un perro.

Le preocupaba mucho la vida del animalito, porque además de que toda la familia profesa un verdadero amor por los animales, este pollito era de su niño, quién sufriría mucho si el pollito moría. 

Le pidió al médico que hiciera lo que fuera necesario; este gesto nos conmovió, porque considerábamos más fácil comprar otro pollo, el cuál no pasaría de 5 pesos. Nos dejó ver la gran ternura que sentía por su hijo y por el animalito.

Este pollito ya estaba algo crecido, le empezaban a salir las plumas blancas, y podía resistir la anestesia. 

Después de revisarlo le explicó el médico que era necesario terminar de amputar el ala, y esa era la peor herida, pero que podría vivir tan solo con un ala.

Así se hizo, y pudimos tener el gusto de verlo recuperarse, ya que lo traían a curación y revisión. 

Una vez que hubo sanado, ya no lo volvimos a ver, hasta que llegó a la edad adulta, y nos invitaron a su casa para verlo.

Sentí mucha satisfacción, él no parecía carente de nada, recorría el corral con la majestad y seguridad de cualquier gallo, y además las plumas impedían ver la ausencia del ala.

Vivió feliz unos cinco años.


\chapter{¡Águilas!}
Un día llegó a la clínica un muchacho portando dos águilas que daban lástima; al parecer, eran hermanas, pero en un estado de descuido tal que no podían tenerse en pie. Estaban muy mal alimentadas, y las patas se habían atrofiado.

El doctor explicó como debía de ser su cuidado y su alimentación, pero ante la complicación, el muchacho decidió mejor dejarlas, pagando 1 semana de cuidados, y quedando muy seriamente de volver al concluir la semana.Volvió, y otra vez la semana siguiente, pero no mas. Una se iba recuperando mejor que la otra, y un día en que la sacaban como siempre al sol, a hacer ejercicios de aleteo y fortalecimiento de las patas, las observábamos con interés mi esposo, mi hijo y yo; ellas estaban en el piso, echadas, comentábamos que ya no había vuelto el dueño, y sobre todo observábamos que la débil estaba peor (Por la atrofia mencionada su posición era abierta de patas, y estas, además, torcidas; la más fuerte estaba siempre de lado). Nos llamó la  atención como la fuerte volteó a ver a su hermana, justo cuando hablábamos de ayudarle a terminar con su sufrimiento. En ese momento el águila fuerte estira su pata sana, clava sus garras en el cráneo de su hermana, matándola al instante. Todo fué tan rápido que no pudimos hacer ningún movimiento. Con asombro vimos cómo, después de esto, se acurrucaba junto a su hermana, como aliviada de que ya no sufriera.

Mi hijo se hizo cargo del águila sobreviviente, llamándola ``Lilith'', que entre los judíos era la primera mujer, antes de Eva. Día a día la alimentaba y le daba rehabilitación. Llegó el momento en que pudo volar bajo, recorriendo todo el patio. Ese día me emocioné mucho, imaginándola dueña del cielo, como debería ser. Pero al fin logró llegar a la azotea, y ahí se quedaba, y solo bajaba cuando mi hijo la llamaba para alimentarla, o para dormir, resguardándola así de los gatos que suelen merodear por las noches. Cada vez volaba más y se alejaba más de la casa, pero al llamarla mi hijo por su nombre, no tardaba en presentarse.

En una ocasión la llamaba, y no aparecía, se subió a la azotea a buscarla, y creyó verla en el edificio de enfrente. La llamaba, y ni se inmutaba; después de gritar su nombre varias veces, lo sorprendió llegando por detrás de él, dándose cuenta, que lo que el veía era un halcón salvaje comiéndose una rata en la punta de una antena de radio. Recuerdo aquel día en que llegó a pararse en el balcón de un conocido monumento arquitectónico leonés arruinado, La Plaza de Gallos, que está enfrente de la clínica. Cómo causaba expectación entre la gente que pasaba, y que orgullo sentía yo, que podía verla llegar tan alto.Cuando mi hijo la llamó, ella inmediatamente obedeció, entrando por la puerta de la calle, maniobra que ponía a prueba su destreza, ya que la puerta es estrecha y sigue un pasillo de unos veinte metros de largo, cuajado de obstáculos, y el cual tuerce en ángulo recto, para volver a torcer 5 mts más allá. Regalándonos, en fin con todo un recital de destreza aeronáutica.


Un viernes, no bajó a dormir; después de llamarla varias veces y buscarla en la azotea, bajó mi hijo, preocupado de que le hubiera pasado algo.A día siguiente, temprano, se presentó con gran algarabía, como diciendo: ``¡Ya llegué, ya estoy aquí''. Nosotros lo interpretamos como que se fué de antro, por que lo volvió a hacer el siguiente viernes, y al otro. Después de algunos viernes, se tomó también los sábados. Después de varias semanas así, también incluyó el domingo, de modo que los fines de semana ya no estaba aquí, pero presentándose muy contenta los lunes, y pidiendo comida. Una ocasión en que no volvió, la llamábamos y  la buscábamos, pero ella no acudía. Una persona vino a decirnos que habían atrapado un águila en un estacionamiento a espaldas de la casa. Mi hijo acudió inmediatamente, y regresando con ella. 


Ya tendría aquí cerca de un año, cuando finalmente no volvió. Tuvieron que pasar varias semanas para que, poco a poco, dejásemos de esperarla\ldots


\chapter{Cada cabeza es un mundo}
Tengo muy claro el recuerdo de aquella vez en que despachaba unos peces, y el muchachito al que atendía me señaló una pecera, pidiéndome un pez, ---``pero que sea macho''---.

Yo traté de aclararle que cuando son ovíparos no se aprecia diferencia, al contrario de los vivíparos, en que se distingue fácilmente el sexo, señalándole una pecera con este tipo de peces, para que observara la diferencia.

Él volvió a preguntar: ---``¿No sabe cuál es macho?''---``No, no se distinguen\ldots''---``Bueno, ni modo, entonces deme una hembra\ldots''---.

Por ese entonces llegaba otro chico más o menos de la misma edad: Unos trece años, pidiendo una tarántula; nos decía que ya había estado juntado dinero, y que ya se la quería llevar.

Nosotros le decíamos que no había, que tenía que esperar hasta el verano.

Pasaba casi diariamente, y en una ocasión le dijimos, a ver si así entendía, que hasta tiempo de lluvias, pero él continuaba pasando diariamente, ahora ya solo desde la puerta gritaba ---``¿Ya llegaron las tarántulas?''---, y nosotros con el dedo le decíamos que no. Pero un día cayó una lluvia ligera y breve, y al poco rato el muchacho llegó derrapando, y preguntando ---``¿Ahora sí ya llegaron?''---\ldots

En cierta ocasión en que el médico escribía la receta para un perro enfermo, le indicaba al dueño como darle los medicamentos y le explicaba que con una jeringa era más fácil por que está graduada, y se puede medir la dosis, y además, si el perro no quiere abrir la boca, por un lado hay un hueco entre los dientes por el que se introducir el extremo de la jeringa.

A los dos o tres días regresó el mismo señor, preguntando si no habría una mejor forma de darle las medicinas, por que así lo picaba mucho con LA AGUJA\ldots

En otra ocasión un señor entró a la clínica, para preguntarle al médico cuál era el horario por la tarde, pues él, siempre que pasaba encontraba cerrado. Mi esposo le contestó ---``De 5:00 a 8:00 pm..\ aquí estamos para servirle''---``No, no, a mí deme una hora fija, ¿como voy a estar esperando para ver si usted llega?\ldots''.

\chapter{Accidentes}

Allá en los albores de la carrera de mi esposo, un tío le pidió que le castrara su puerco. Él se rehusó diciendo que su especialidad eran las pequeñas especies, pero su tío no escuchaba esto. Solo decía que él era veterinario y que podría hacerlo.

Mi Gordo aceptó, pensando que quizá fuera sencillo, y nos trasladamos a la colonia \textit{Gabriel Hernández}. 

Ya estando en el domicilio, Tony preguntó entre sus sobrinas quién querría ser su ayudante, y eligió a una, diciéndole que ella se haría cargo  de la mascarilla de cloroformo, la cual aplicaría y retiraría cada determinado tiempo.

Al llegar a la parte más importante de la cirugía, el doctor se concentró en su trabajo, y la sobrina también, pero en el mismo. De pronto al galeno le llamó la atención que no sangrara la herida, volteó a ver a la sobrina, y la sorprendió absorta en lo que él hacía: Se había olvidado de la mascarilla, y el puerco había muerto.

Intentó reanimarlo, pero no lo logró, ante lo cual nos sentimos muy avergonzados, pero estoy segura que yo más, sobre todo ante las palabras de su tío, que le dijo: ---``Yo creí que si sabrías hacer esto tan simple''---. De todas formas nos invitaron para el día siguiente a las ``carnitas''; yo hubiera preferido no ir por pura vergüenza, pero, para mi esposo tan sociable, lo ocurrido había sido ``pecatta minutta'', así que asistimos, y de esa manera, me enteré de la forma en que se preparan las ``carnitas''. Además, comentábamos divertidos el incidente del día anterior, quedando borrada de esta manera la vergüenza que yo sentía. Por cierto que quedaron deliciosas esas ``carnitas''.

En otra ocasión le trajeron un \textit{doberman} enfermo, pero no se dejaba examinar. Ante el nerviosismo del animal, el dueño dijo que él lo controlaría, que a él no le mordería. Metió la cabeza del perro bajo su brazo, quedando así inmovilizado, pudiendo así el médico hacer muy bien todo su trabajo, y al terminar, le dijo al señor que ya lo podía soltar. Cuál sería la sorpresa, el perro cayó exánime, pues su dueño había apretado demasiado fuerte\ldots


A nuestra hija menor le gustó la estética canina. Mucho aprendió de su papá, además de tomar cursos especializados. Lo hacía rápido y bien, y guardando lo que ganaba pudo un día comprarse su coche.

Pero hubo un día en que le trajeron una perrita poodle, demasiado nerviosa, pero no agresiva. Esta perrita ya había venido varias veces, de modo que mi hija comenzó su labor confiadamente. Cuando ya había hecho la mitad del trabajo, de pronto la perrita se desvaneció: Al acudir el médico, se dio cuenta de que había sufrido un paro cardiaco. 

Muy angustiado, se preguntaba como informar al dueño, con el que había una cierta amistad; cuando éste llegó y supo lo que había pasado, tuvo una reacción totalmente inesperada: Le dio la mano al médico muy agradecido, explicándole cómo, de esta manera, se solucionaba un problema familiar, ya que la perra les impedía realizar un viaje a Europa, naturalmente muy deseado. El alivio fué general\ldots

\chapter{La peor experiencia}

La parte más difícil para un veterinario es tener que dormir a sus pacientes, pero en lo que voy a narrar en seguida este sentimiento inmediatamente desapareció para dar paso a otros mucho más fuertes e intensos.

Una señora se presentó en la clínica preguntando cuanto le cobraría por sacrificarle tres gatos. El médico le preguntó por qué había tomado esa decisión, y la señora contestó que se habían vuelto agresivos, y que cada vez que pasaba, la rasguñaban o la mordían, y prácticamente ya se habían apropiado de la planta baja, teniendo ella que recluirse en el piso de arriba.

No fué fácil para el médico aceptar este trabajo, pero por fin acordaron un precio. La señora pregunta: ---``¿Y si fueran cinco gatos?''---``Sería tanto''---``¿Y si fueran ocho?''---, a lo que el médico le pregunta: ---``Oiga, ¿es un juego?''---, a lo que ella responde ---``No doctor, es que no se cuantos son, tal vez sean quince''---. 

Fijaron el día y la hora en que se presentaría el médico en el domicilio de la señora.

El día acordado, el médico dispuso el equipamiento necesario, y le pidió ayuda a nuestro hijo mayor, porque calculaba (muy por debajo) lo que le esperaba. 

Al llegar al lugar, comenzó el horror con el olor que llegaba hasta la calle. Al ser introducidos en la casa, fué subiendo de tono este horror, al ver el piso cubierto de diarrea, y sintiéndolo resbaladizo.

La señora se subió a sus habitaciones, y quedaron ellos sin saber por donde empezar. Los animales sienten o presienten, y en este caso comenzaron a maullar agresivamente, pudiendoseles ver y escuchar por todas partes: Entre los cojines del sofá, por debajo del mismo, atrás y encima del refrigerador, sobre el aparador, las vitrinas, por todos lados. 

Ellos se pusieron guantes de electricista, y llevaban un bastón especial para dominar animales agresivos, pero no esperaban tan formidable ataque.

Comenzaron por despachar a los más cercanos, y otros se les lanzaban encima furiosamente; ellos los recibían con golpes del bastón, o del maletín etc.

Era una lucha a muerte; la adrenalina de ambos lados llegó a niveles muy altos, sudando por el esfuerzo, el miedo profundo, y el choque emocional, ya que nunca se imaginaron la situación en la que se encontraban.

Además, a este horror, se le sumaba la presencia de miembros muy jovenes, tanto que todavía no terminaba su desarrollo, y dolorosamente, también había hembras preñadas.

Esto ya no era cuestión de cumplir con un trato, sino de la seguridad de la señora, qwe ya era mayor y además vivía sola. No hubiera sido la primera vez que los gatos terminaran devorando a su propietaria.

Sintiéndose hastiados de muerte, los valientes hombres iban acumulando los cadáveres, que llegaron a 28, con los músculos entumecidos, buscaban con pavor si había mas. Solamente uno logró escapar, colándose por un pequeño orificio, y al salir en su persecución a la calle, ya tan solo lo alcanzaron a ver a lo lejos, la cola tensa como una antena, y la velocidad que llevaba les hacia pensar que no iba a detenerse nunca\ldots

Ha sido, con mucho, la peor experiencia en 45 años de ejercicio profesional.

\chapter{¿Las mascotas entienden nuestro idioma?}
Hace pocos meses llegó al consultorio una clienta que acudía con regularidad trayendo dos perritos criollos muy tranquilos, y esta vez además, traía una perra \textit{pitbull} ya crecida que estaba sentada junto a su ama, esperando su turno.

Al salir Tony a recibirlos, le preguntó inmediatamente si era suya la perra, a lo que contestó que sí. El médico comenzó a decirle que no le convenía tenerlos juntos, pues tarde o temprano los agredería, y quizás mortalmente; la dueña como que no creía que fuera para tanto, y él siguió explicando cómo era el temperamento de esta raza.

De pronto, la perra, sin quitarle los ojos de encima, se levantó y mordió al doctor en un muslo, y regresó junto a su ama, la cual estaba sumamente apenada por este percance, y aseguraba que no tendría cara para volver, lo cual ha cumplido, y no sé si este incidente fué bastante elocuente para ella, y tampoco sé si conserva a la perra o no, el caso es que en el muslo de mi esposo todavía hay huella de la mordida y me pregunto, si la perra entendió lo que decía, o si captó más el tono de voz y la gesticulación, pero el doctor asegura que, basado en su experiencia, la perra entendió. ¡Claro que sí!.

Teníamos una pareja de perritos \textit{chihuahua}; el machito llegó a nosotros de mes y medio, y después de un tiempo llegó la hembrita, de aproximadamente 1 año, muy maltratada y traumatizada emocionalmente. Se quedó con nosotros, porque su dueño pidió que se le sacrificara, ya que no quería gastar en vacunas, y sinceramente no le gustaban los perros. El médico le dijo que no podía hacer eso, que la perrita, a pesar de su estado, estaba muy joven, y su estado físico y emocional solo se debían al descuido, que solo necesitaba una atención adecuada y cariño. Por mi parte me alegró mucho, pues era muy bonita. Era blanca con manchas color champagne muy tenues, y cuando supe que necesitaba cariño, pensé que yo podía encargarme de eso. Era más esbelta que el machito, el cual tenía un hermoso color cobre, con el pecho, guantes y botas blancos. De inmediato congeniaron, pero se veía que ella dominaba, ya que comenzó a tomar la costumbre de sentarse sobre el, cuando el tomaba su siesta. 

Le dimos mucha atención, pero ella siempre demostraba mucho miedo, y muy mal apetito. Probamos todos los aderezos apetitosos para perros, abre-apetitos y complementos alimenticios, pero no conseguimos terminar con este problema. 

Acostumbraba, tomar una croqueta, una sola, la dejaba en otro lugar, y después de mucho rato se la comía. Aún así se embarazó, y fué muy vigilada su gestación, por el temor que causaba su desnutrición.

Un sábado por la noche, estando mi hijo mayor solo con los perritos, veía la televisión, cuando entró el machito, el cual se sentó ante el, mirándolo fijamente. Por su mirada comprendió al instante que pasaba algo, buscó a la hembrita, y ciertamente estaba en trabajo de parto. Avisó a su papá, el cual de inmediato le dio atención. 

Esa noche nacieron dos cachorritos, uno vivo y uno muerto. Aquel, al día siguiente también murió. Dijo el médico que había muerto por inhalación de meconio. 

Era una tristeza ver a la pobre mamá buscando a sus hijos. Entraba en un lugar y en otro, con los ojos llenos de ansiedad, como interrogándonos.

\chapter{La guacamaya}

---``Dice el doctor que si le revisa este animalito, por favor, que \textit{ahorita} viene él''---. Casi de inmediato llegó el afamado médico cirujano vecino nuestro, que acababa de adquirir una guacamaya roja, muy chica de edad, como de dos meses. Apenas le estaban empezando a salir las plumas, y por supuesto no sabía comer sola. 

La primera pregunta del doctor, fué sobre el valor de las guacamayas rojas.Mi Gordo esquivó la pregunta, porque notaba algo que no encajaba: Estaba muy húmeda, y dejaba las manos pintadas de rojo. Lo que había pasado era evidente: La habían teñido con tinte para el cabello. El veterinario le devolvió la pregunta: ---``¿Pues a como te la dieron?''---. La respuesta: ---``Me la dieron en pago por una cirugía''---. El veterinario dice que la cuestión es que NO ES ROJA, ---``Está pintada, ve mis manos''---. Lo que estamos viendo es una variedad de la guacamaya militar, la cual es verde. El doctor quiso saber cuanto podría costar; la respuesta fué que tal vez una quinta parte. Al escuchar esto, el doctor dijo ---``Ah, pero han de volver, pues falta el seguimiento de la cirugía''---.

En seguida mi esposo le informó que estas aves son dependientes de sus padres por mucho tiempo, y que su manutención requiere mucho cuidado: Le dijo que debía prepararse una papilla a base de frutas, y dársela cada cuatro horas, introduciendo el bocado hasta el fondo de la garganta, pues a esta edad ni siquiera eso pueden hacer solos.

Al referido doctor le pareció muy complicado, y decidió dejarla con el veterinario un tiempo, el cual se prolongó por tres meses. 

Al principio no abría el pico; mi esposo se lo abría y le daba su papilla con una jeringa grande, adaptada para este caso. Después de varios días la guacamaya ya comenzó a abrirlo, y a agitar las alas y a gritar en cuanto lo veía, pues lo consideraba su madre\ldots

A nosotros nos causaba pena ver a un animalito tan indefenso, separado de sus padres a tan temprana edad. 

Mi esposo fué modificando la dieta de acuerdo a los requerimientos del desarrollo del animalito: Agregaba a la papilla diferentes insectos, semillas y fruta seca. 

Los gritos demandantes de la guacamaya habían ido cambiando con los días, ganando en timbre y potencia. Su aspecto también iba variando con la salida de las plumas. Hubo un momento en que tenía apariencia de \textit{confetti}: El fondo rojo, la mayor parte de las plumas verdes, algunas azules y otras amarillas. El doctor vecino y su familia venían a visitar a su guacamaya, esperando verla comer sola, pero todavía faltaba mucho tiempo para que esto se lograra. 

La esposa del doctor vino dos veces para aprender como cuidarla, aunque controlando su miedo, ya que le asustaban mucho sus poderosos gritos. Finalmente se animó a llevársela, y no la hemos vuelto a ver; solo nos dice el doctor que se ha desarrollado muy bien, y que está preciosa, aunque los despierta muy temprano con sus estridentes gritos. Hemos pensado pedirle al doctor que le tome una fotografía, porque deseamos mucho verla convertida en adulto.

\chapter{¡Somos abuelos!}
Nuestros tres hijos ya eligieron camino, solo nos quedamos con el menor que tiene quince años. Mi vida es muy activa, y me agrada como es, seguimos bailando mucho, sentimos una verdadera pasión por el baile.  

Es septiembre de 1996 y estoy al cuidado de la tienda de animales y accesorios, mi esposo se ha ido al congreso veterinario que se realiza en la primera semana de este mes, y es el 2º en su historia. De pronto suena el teléfono, y al contestar me dice Tony que si quiero una competencia de baile de salón; aterrorizada, pregunto ---``¿Cuando\ldots?''---, él me dice ---``Hoy''---, pregunto ---``¿A que hora\ldots?''--- y él me dice: ---``¡Ya!''---.

La adrenalina comienza a correr por micuerpo y quiero decir que no, pero le pregunto si lo desea mucho, él contesta que sí. Pienso que no tengo derecho a privarlo de esta experiencia, y cerrando los ojos acepto. Será la primera vez que competimos.

Viene por mí, y al llegar al Centro de Convenciones, veo la cantidad de personas que están esperando el concurso. Vuelvo a sentir terror y le pregunto que qué le hace creer que podemos tener alguna oportunidad; que no por el hecho de que toda nuestra vida nos ha gustado el baile, quiere decir algo mas, fin, mi terror aumenta al ir a inscribirnos, y veo que todas las parejas son jóvenes, nosotros somos los únicos abuelos. Mi esposo procura tranquilizarme, diciéndome que va a estar muy fácil, que se trata de baile de salón. ¡Que horror!. El templete donde bailaremos está como a metro y medio de altura; nos dicen que ya tenemos nuestro lugar, y no me queda más que poner buena cara y sonreír. 

Ignoro a las 3000 personas que están alrededor, y al empezar la música comienzo a disfrutar como pez en el agua. Van eliminando parejas, y nosotros continuamos. Cuando quedamos seis parejas nos indican que salgamos, lo cual hacemos, pero el público comienza a gritar, protestando:---``¡Ellos no, ellos no!. ¡Voto por voto, casilla por casilla!''--- Y nos regresan. Al final quedamos en 3er lugar, nos vencieron con ``reaggeton''. El premio es un aparato de DVD cada uno, y al llegar a casa, nuestros hijos ya están enterados y nos felicitan. El yerno mayor nos pregunta que vamos a hacer con esos aparatos, contestamos que no sabemos, agregamos que el nuestro está casi nuevo. Él nos da 2000 pesos por cada uno diciendo que ya tiene a quien vendérselos. Me siento muy satisfecha porque después del pánico gané algo de dinero; mi esposo me dice tiernamente que confíe más en mí. Ha sido una emoción tan fuerte, que me impide dormir pronto.

Me he inscrito en natación; me aterroriza cuando el agua me cubre la cara, y creo que sabiendo nadar voy a poder controlar ese miedo, además he leído que es un deporte muy completo.

Ya han pasado tres meses y todavía no se nadar, no se ocupan de mí, lo poco que he logrado ha sido por mi cuenta, observando las indicaciones que dan a otras mujeres.

Me han contado que al occidente de esta ciudad inauguraron una alberca techada y con agua tibia, y pertenece a una deportiva munincipal; convenzo a mi esposo y nos inscribimos. Aquí sí voy adelantando; lo más hondo está enmedio, y al pasar por ahí no dejo de ponerme nerviosa, pero quiero anotarme este reto.

A Tony le está sirviendo mucho para las varices, y a mí para mi auto-estima.

Ya ha pasado un año, nuestros esfuerzos y tenacidad están dando sus frutos, pues ya podemos nadar en los cuatro estilos y nos han puesto una entrenadora excelente, (a quien llamamos ``Maru''), con quien llevamos amistad, y ella nos hace sentir su aprecio.

¡Otra vez, más adrenalina!: Quiere Maru que participemos en una competencia.

Tony se ha safado, diciendo que debe atender su trabajo, y a mí no me queda más que aceptar, pues nos ha dicho ella que le exigen determinado número de personas. ¡Ay, no! A mí no me gustan las competencias, me estreso mucho y no me gusta sentirme así. Me gusta mucho nadar a mi ritmo y sin presiones.

Es sábado, día de la competencia. ¡Que horror! La gradería está llena y yo no quiero participar en esto. 

¡No lo puedo creer! Después de haber aprendido a respirar, me toca mi turno, y un entrenador me va vigilando con un cronómetro, estoy tan nerviosa que casi me ahogo, pienso que ya perdí y en cuanto termino me salgo y me escurro hacia los baños, ya no quiero saber nada. Me estoy vistiendo y escucho que me llaman por el altavoz, yo  sonrío para mí misma y pienso: ---``Que vayan mucho\ldots''---.  Ya vestida, me cuelo hacia el coche y me voy a mi casa. Cuando regrese, pues a ver que invento.

Regreso el lunes, y para mi sorpresa, Maru me entrega una medalla, (``dizque me la gané''), preguntando qué había pasado conmigo, a lo que contesté que me sentía mal, que se me dificulta mucho la respiración, que el ``estress''\ldots.

Ahora que ya estamos casi solos me puedo dedicar más tiempo para mí misma, y como nos gusta viajar, pues ya podemos hacerlo más, pues ya no tenemos colegiaturas ni otros gastos. Disfrutamos de nuestra compañía mutua.

Me veo en el espejo y observo cuanto deterioro ha dejado el paso del tiempo, observo como ha ido cambiando nuestro cuerpo, el cabello de Tony ahora es platinado, pero a mí me sigue pareciendo muy guapo y atractivo.

Yo, a pesar de mi edad, no me siento vieja, gozo de buena salud, y tengo más equilibrio en mi vida emocional, quiero decir que he aprendido a vivir en paz, esto hace que mi vida sea muy agradable; la convivencia con Tony hace que me sienta bonita, valiosa y muy amada.

\chapter{Aprendizaje}
Mientras nuestros hijos son niños, nos adherimos a Encuentros Matrimoniales, agrupación que promueve la buena relación en el matrimonio y la superación de la persona individualmente.

Deseamos fervientemente ser mejores en todos aspectos y dar a nuestros hijos lo mejor de nosotros, esperando que les sirva para su vida independiente.

Somos un grupo de matrimonios que luchamos por lo mismo y nos esforzamos apoyándonos unos a otros.

En esta etapa he ido tirando costumbres y conceptos que ya no tienen valor, y me he ido sintiendo, más libre y más ligera;
además he ido aprendiendo poco a poco cómo quitar las telarañas de mi matrimonio y cómo conservarlo fresco.

Aprovechamos esta experiencia para nosotros mismos y también tratamos de ayudar a otros matrimonios siendo expositores de nuestras propias experiencias.Nos quedamos en este grupo 5 años, sintiéndonos un poco más grandes;doy las gracias a todos los que nos ayudaron.

Al llegar nuestros hijos a la adolescencia, nos convertimos en algo así como papás de muchos muchachos y muchachas de una edad entre quince y dieciocho años, ya que estamos a cargo de la \textit{Onda Juvenil Cristiana}, agrupación de la iglesia que se ocupa de esta hermosa y peligrosa edad; tratando de reforzar valores, ideales, auto-estima, etc. Es conmovedor encontrar tanta nobleza entre estos niños jóvenes. Esto nos ocupó 3 años.

Cuando nuestros hijos ya son jóvenes, cambiamos de ocupación, ahora impartimos en la parroquia charlas pre-matrimoniales, esforzándonos por ayudar a las parejas a estar bien conscientes y seguras del paso que van a dar y con quién lo van a dar.

Todas estas tareas la hemos realizado después de nuestro trabajo, también en sábados y domingos, dejándonos un gran aprendizaje.

Estas ocupaciones han llenado tan por completo mi mente que no he sentido correr el tiempo. De pronto tengo 42 años y ya tengo un pequeño nieto.

Cada uno de nuestros cuatro hijos han acudido al llamado de la vida, y de uno en uno se han ido volando del nido, dejándonos solos, pero no vacíos ni tristes; nos tenemos uno al otro, y todo el tiempo para nosotros.

\chapter{¡Ya tenemos 60!}
Ya llegué a los 60 años, Tony llegó hace dos años, inmediatamente fuimos por nuestra credencial de INAPAM, ya podemos viajar en autobús por mitad de precio, realizamos mejor nuestro proyecto de viajar mucho. ¡Nos da tanto la vida!

Damián también ya se casó, ahora sí ya estamos solos, pero no nos ha afectado, nos tenemos mutuamente. Ya contamos con diez nietos:¡Que gusto!  

Hace algún tiempo que formamos parte de una academia de baile de salon, a la cual asistimos por la noche, después de cerrar la clínica.

En cuanto cumplo los 60, vienen a la academia haciendo una invitación para concursar en baile; nadie acepta, solo nosotros. Nos dan el día y el lugar, a donde llegamos a tiempo llevando nuestro vestuario para los ritmos considerados como ``baile de salón'': Cha-cha-cha, mambo, danzón, rocanrol, swing, etc.

Hemos quedado como finalistas; ahora representamos a León en Guanajuato.  Apenas lo creo, me siento muy emocionada, me cuesta trabajo concentrarme en mis tareas\ldots Vuelve mi inseguridad, porque pienso: ---``Para que vamos a Guanajuato, son muchos los municipios que van a participar y es difícil sobresalir\ldots''---. Tony nuevamente me contradice en esto. 

Ya llegamos a Guanajuato, y ni siquiera admiro la ciudad como suelo hacerlo porque estoy muy estresada. Hemos llegado al lugar y veo que se está congregando las parejas representantes de los municipios.

¡Otra vez quedamos en 1er.\ lugar! Siento que esta realidad no cabe en mi ser, temo que me desborde. Ahora iremos a México a representar a Guanajuato: ¡Que honor y que responsabilidad!

Vamos a México con los gastos pagados y con todo el vestuario, y al llegar a inscribirnos nos dicen que debemos elegir solo 3 ritmos porque no hay tiempo para cambiarnos. Hemos elegido Rock `n roll, mambo y danzón.

No me siento cómoda, no me gusta estar en este lugar y no quiero repetir esta experiencia, ya hablé con Tony desde León, y prometió que no me lo volvería a pedir. 

A pesar de mi malestar hemos quedado en 1er.\ lugar en Rock `n roll, 2º en Mambo y 4º en Danzón (este ya no cuenta, solo los tres primeros), así que cada uno llevamos dos medallas y un trofeo. Tony les avisó a varios amigos que tenemos en México y nos fueron a ver; escuché sus vítores y pude sentir su apoyo. Al concluir el evento nos dirigimos al hotel que teníamos ya pagado, pero al estar cenando decidimos regresar a León en seguida y así lo hicimos, y al llegar acá, nos enteramos que la noticia ya había circulado en los periódicos vespertinos, por lo que al día siguiente tuvimos muchas llamadas para felicitarnos.

Pocos días después el periódico el Sol de León nos pidió una entrevista, dedicándonos una plana con varias foto y una reseña.
Nuestra vida está llena de emociones muy fuertes que a mi me parece mucha y Tony dice que quiere más.

Actualmente, quizás por la edad, quizás por razones de horario, nos hemos acomodado en el DIF, que está a 3 cuadras de nuestra casa. La atracción primera fué que tiene dos áreas para ejercicios físicos y además es un recinto arbolado muy bonito que se encuentra en el conocido Jardín de San Juan de Dios. Después hemos ido decubriendo que imparten talleres interesantes. Nos gusta participar en las fiestas que aquí se organizan, dizfrazandonos de acuerdo al evento, participar en concursos de trajes típicos, concursos de baile, etc.
Hoy en día formamos parte del taller de teatro, y ya estrenamos nuestra primera obra, también estamos en el coro, el circulo de lectura y mas.

Por el momento ahí estamos plantados, al mismo tiempo atendemos la clínica, aunque ya no tenemos tienda de animales, la clínica funciona muy bien, mi esposo se siente afortunado de poder hacer lo que más le gusta, y ademas le pagan.

Me gusta lo que me ofrece la vida en esta etapa, más tiempo para mí y como pareja. Puedo leer mucho, escribir, viajar y hacer casi todo lo que me gusta. También seguimos participando de muchos concursos de baile: ¡Pos' ¿no que no?!

¡Que felicidad, ya tenemos nuestra primera bisnieta!

En total, me gusta mi vida como es; la  considero \textsc{una vida plena}.
\end{document}
